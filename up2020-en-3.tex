\newpart
\section{From Lisp to Scheme}\label{sec:lisp}

After the discussions about minimalism outside programming in last part, now
we come back to the subject of programming; but before examining Unix itself,
let's first discuss something outside the Unix circle -- the Lisp family of
programming languages\cupercite{wiki:lisp}.  Born in 1958, Lisp is one of the
oldest programming languages, along with Fortran (1957), Algol (1958) and Cobol
(1959), and is perhaps the only one of them still constantly attracting new
interest as of now: Fortran finds little (and still diminishing) use outside
of numerical computation in science and engineering, few young programmers are
interested in maintaining existent (stale and also diminishing) Cobol codebases,
and it is fair to say that Algol as a language has been dead for long.  So what
is the reason for the continued vitality of Lisp?  The answer is Lisp's inherent
simplicity, and its strong \stress{expressiveness} despite the simplicity;
in this section, we will briefly examine the simplicity of Lisp.

Lisp was designed by John McCarthy\footnote{By the way, he was also one of the
pioneers in time-sharing operating systems.}, who showed that a practically
usable Turing-complete language can be built with a handful of primitive
functions in addition to conditional expressions and recursive function
definitions\cupercite{mccarthy1960}.  Steve Russell noted, to McCarthy's
surprise, that the \verb|eval| function in Lisp could be directly translated
into assembly code, and the result was the first Lisp interpreter, running on
an IBM 704 mainframe, compiled manually by Russell.  Alan Kay, the designer of
Smalltalk, made the following comment\cupercite{feldman2004} about the idea:
\begin{quoting}
	\emph{[...]} that was the big revelation to me when I was in graduate
	school -- when I finally understood that the half page of code on the
	bottom of page 13 of the Lisp 1.5 manual\cupercite{mccarthy1962} was Lisp
	in itself.  These were ``Maxwell's Equations of Software!''  This is the
	whole world of programming in a few lines that I can put my hand over.
\end{quoting}
Although Lisp can be reduced to a minimal core, not all languages in
the Lisp family are indeed minimalist: for example, now the unqualified
name ``Lisp'' often means Common Lisp, which is fairly big even in terms
of core functionalities; nevertheless, proper abstractions to compress
boilerplate code are still strongly encouraged by Common Lisp experts,
like Paul Graham\cupercite{graham1993}.  However, there does exist
a truly minimalist Lisp, and it is Scheme.

Scheme was not intended to be minimalist from the beginning, according to its
designers, Gerald Jay Sussman and Guy L.\ Steele Jr.\cupercite{sussman1998}%
\ (note how similar this is to Unix's conscious pursuit for simplicity after
the inception of pipes -- \cf~Section~\ref{sec:mcilroy}):
\begin{quoting}
	We were actually trying to build something complicated
	and discovered, serendipitously, that we had accidentally
	designed something that met all our goals but was much simpler
	than we had intended.  \emph{[...]} we realized that the
	\stress{$\bm{\lambda}$-calculus} -- a small, simple formalism --
	could serve as the core of a powerful and expressive programming language.
\end{quoting}
Anyway, minimalism became a defining property of Scheme, as can be seen
from the first sentence from the introduction to its specifications,
\emph{Revised$\,^3\!$ Report on the Algorithmic Language Scheme} and
onward (usually called \rnrs{3}, \rnrs{4} \etc)\cupercite{scmreports:home}%
\footnote{\label{fn:r6rs}The specifications up to \rnrs{5} were all a few
scores (usually less than 60) pages of text, including examples and rationales;
\rnrs{6} significantly expanded both the core and standard libraries\cupercite%
{weinholt2019}, in order to standardise some (more or less legitimately)
demanded features present in mainstream languages.  \rnrs{6} was highly
controversial, most importantly because certain features are less useful
for educators, hobbyists \etc{} who use just a small subset of the
functionalities\cupercite{scmreports:position}; as a result, \rnrs{7}
was intentionally separated into a ``small'' specification and a
``large'' one.  Incidentally, the specification of Go is also
rough 50 pages, but Go is much less expressive than Scheme
(\cf~Section~\ref{sec:homoiconic} for an example).}:
\begin{quoting}
	Programming languages should be designed not by piling
	feature on top of feature, but by removing the weaknesses and
	restrictions that make additional features appear necessary.
\end{quoting}
This attitude clearly resembles the Unix philosophy, and it makes Scheme
the most expressive programming language in my opinion; for this reason,
in all following text in this document I will use Scheme to represent Lisp.

The reason for this digression about Lisp is its strong expressiveness despite
the simplicity, and this expressiveness largely comes from the way Lisp code is
written -- \stress{S-expressions} (short for symbolic expressions)\footnote%
{In the original paper for Lisp\cupercite{mccarthy1960}, McCarthy actually
intended to use another form called M-expressions (short for meta expressions)
for source code, and let the computer translate them to S-expressions; however,
programmers generally preferred to use S-expression directly.  M-expressions are
homoiconic just like S-expressions, and a format closely related to the former
is used by Mathematica, the computational software.}; for example, the factorial
function can be written in Scheme and Python respectively as\footnote{They
are not exactly equivalent: first, Scheme does not have a \texttt{return}
primitive, and instead use expressions in \stress{tail positions} as well as
\stress{continuations} (\cf~Footnote~\ref{fn:cps}) to return values; second,
\texttt{(define (fn (args ...)) ...)} is a shorthand for \texttt{(define fn
(lambda (args ...) ...))} in Scheme, while Python's \texttt{lambda} is
severely limited (\cf~Section~\ref{sec:homoiconic} for an example).}:
\colskipa\begin{multicols}{2}
\begin{quoting}
\begin{Verbatim}
(define (f i)
  (if (> i 1)
      (* (f (- i 1)) i)
      1))
\end{Verbatim}
\end{quoting}
\begin{quoting}
\begin{Verbatim}
def f(i):
  if i > 1:
    return i * f(i - 1)
  return 1
\end{Verbatim}
\end{quoting}
\end{multicols}\colskipb\noindent%
S-expressions would probably appear unintuitive to a newbie, but they are in
fact much easier for computers to process than most alternative source code
formats, and still quite readable to the programmer after a small degree of
familiarisation.  You may note that this is very similar to why text streams
are used as the interface for traditional Unix tools (\cf~Section~%
\ref{sec:mcilroy}), so why are S-expressions more machine-friendly, and
what technical advantages do they bring us?  Please read the next section.

\newpart
\section{S-expressions and homoiconicity}\label{sec:homoiconic}

To understand the machine-friendliness of S-expressions, we first need to
know how source code for a program is represented internally by the computer:
whatever language you use, the source code will be converted (\stress{parsed})
into a data structure that represents its syntax elements necessary for the
computer to understand the program, and this data structure is called the
\stress{abstract syntax tree}\cupercite{wiki:ast}.  For instance, the
AST corresponding to the expression $4 \times 2 - (a + 3)$ and
the S-expression equivalent to the latter are shown below:
\colskipa\begin{multicols}{2}
\begin{quoting}[innerleftmargin = 0.45em,
	innertopmargin = -0.15em, innerbottommargin = 0.15em]
\begin{forest}
	for tree = {l sep = 0pt, l = 2.25em}
	[$-$,
		[$\times$, [$4$] [$2$]]
		[$+$, [$a$] [$3$]]]
\end{forest}
\end{quoting}
\columnbreak\vspace*{-0.64em}
\begin{quoting}
\begin{Verbatim}
(-
  (* 4 2)
  (+ a 3))
\end{Verbatim}
\end{quoting}
\end{multicols}\colskipb\noindent%
S-expressions can be regarded as representations for nested lists, and
as we can see from the example above, nested lists map to tree structures
(including ASTs) naturally; additionally, although S-expressions do not
eliminate the need for parsing, they are technically much easier to parse
than other formats, which makes them a most convenient representation for
ASTs.  Therefore in combination with the strong capability for list
manipulation in Lisp (originally ``LISt Processor''), S-expressions enable
Lisp to almost effortlessly process Lisp code just like regular data.

When a language can modify source code written in itself just like other
data, it is called a \stress{homoiconic} language\cupercite{wiki:homoiconic};
the ability to treat code as data has extremely profound implications,
as can be seen from the beginning of the half-joking document \emph{The
Gospel of Tux}\footnote{Apart from the von Neumann Architecture\cupercite%
{wiki:neumann}, the profundity of the ``code is data'' idea can be
traced to earlier origins, for instance the Gödel numbering\cupercite%
{wiki:godelnum} and the halting problem\cupercite{wiki:halt}.}, which
is why I consider homoiconicity a notion as profound as complexity:
\begin{quoting}
	In the beginning Turing created the Machine.  And the Machine was crufty
	and bogacious, existing in theory only.  And von Neumann looked upon
	the Machine, and saw that it was crufty.  He divided the Machine into
	two Abstractions, \stress{the Data and the Code, and yet the two were
	one Architecture}.  This is a great Mystery, and the beginning of wisdom.
\end{quoting}
It was mentioned in Section~\ref{sec:boltzmann} that all Turing-equivalent
models of computation are not equally wieldy for generic discussions, and
similarly all homoiconic languages are not equally wieldy for practical ``code
as data'' usage\footnote{Another example is XML, which is a descendant of SGML,
which in turn is a document markup language, and this makes the XML markup quite
verbose.  Although XML naturally models a tree structure, its use outside of
document markup often results in files with more markups than contents; in
addition, the complex specification of XML makes it nontrivial to parse, and
therefore unsuitable as source code for programs.}: for example, it might be
said that all languages which support text processing are in a sense homoiconic,
because the source code is text; however, most \stress{code transformations},
that are interesting to implementers of interpreters and compilers, would be
very cumbersome to implement only with text processing mechanisms.  For this
reason, in all following text in this document, when discussing homoiconicity
I will specifically mean it on the \stress{AST level}, or in other words
the ability to easily modify the AST of source code just like regular data.

In a language with AST-level homoiconicity, the macro system can be implemented
to use functions, that transform ASTs, as macros; with this kind of macros, new
syntaxes can be easily defined, and many \stress{language features} that are
hardcoded in non-homoiconic languages can be implemented as elegant extensions
to the core language.  For example, goroutines, which are usually considered
one of the main features of Go, are essentially \stress{syntactic sugar} for
interfaces to lightweight coroutines in the libthread\cupercite{lucent2002}/%
libtask\cupercite{swtch:libtask} model\footnote{\label{fn:cps}Another model is
based on the \stress{continuation-passing style}\cupercite{wiki:cps}.}, and
they are hardcoded in Go exactly because new syntaxes cannot be easily added
to Go.  The example about goroutines might seem slightly abstract because
we have not seen code transformations in action, and here I give
an example that is both more concrete and more advanced.

Anonymous functions, essentially $\lambda$-expressions, are very useful
when used as function arguments, as is shown in the Python code below:
\begin{quoting}
\begin{Verbatim}
sorted(l, key = lambda e: e[0])
\end{Verbatim}
\end{quoting}
However, the function body of \verb|lambda| in Python must be an expression,
so assignment cannot be used, for instance the following function on the left
cannot be directly transformed to a \verb|lambda|.  But if we regard assignment
as a kind of filter that converts a tuple of input values (\eg~$(x, y)$) into
a modified tuple of values (\eg~$(x, y')$), the function can be transformed
into the composition of multiple \verb|lambda|s, as is shown on the right.
\colskipa\begin{multicols}{2}
\begin{quoting}
\begin{Verbatim}
def f1(x, y):
  y = y / (1 - x)
  x = x + y
  return x * x + y
\end{Verbatim}
\end{quoting}
\begin{quoting}
\begin{Verbatim}
f2 = lambda x, y:
  ((lambda v: v[0] * v[0] + v[1])
  ((lambda v: (v[0] + v[1], v[1]))
  ((x, y / (1 - x)))))
\end{Verbatim}
\end{quoting}
\end{multicols}\colskipb\noindent%
The definition of \verb|f2| is evidently ugly, and part of the ugliness
lies in the explicit value tuple \verb|v|; another source of ugliness is the
reversed order of filters -- the last \verb|lambda| to be applied is written
first, because it is the outermost.  In Scheme, the \verb|let| form has
its input values written before its body, so we can use it to compose
filters in the desired order, as is shown below on the left.
\colskipa\begin{multicols}{2}
\begin{quoting}
\begin{Verbatim}
(define f3
  (lambda (x y)
    (let ((y (/ y (- 1 x))))
    (let ((x (+ x y)))
    (+ (* x x) y)))))
\end{Verbatim}
\end{quoting}
\begin{quoting}
\begin{Verbatim}
f3 = lambda x, y: \
  (lambda y:
    (lambda x: x * x + y)
    (x + y)) \
  (y / (1 - x))
\end{Verbatim}
\end{quoting}
\end{multicols}\colskipb\noindent%
As you might have already guessed, \verb|let| is just syntactic sugar for
\verb|lambda|, and \texttt{(let ((a \emph{e$_1$}) (b \emph{e$_2$}) ...) ...)}
is a shorthand for \texttt{((lambda (a b ...) ...) \emph{e$_1$} \emph{e$_2$}
...)}, so \verb|f3| is equivalent to the Python function above on the right.
You should have also noticed \verb|v| is absent from \verb|f3|, and this
works because arguments of inner \verb|lambda|s (\eg~the \verb|x|
in \texttt{(lambda x: x * x + y)}) shadows the outer variables
with the same names (\eg~the argument \verb|x| for \verb|f3|).

\colskipa\begin{multicols}{2}
\begin{quoting}
\begin{Verbatim}
f4 = lambda x, y: check(
  divide(y, 1 - x),
  (lambda y:
    (lambda x: x * x + y)
    (x + y)))
\end{Verbatim}
\end{quoting}
\begin{quoting}
\begin{Verbatim}
check = lambda arg, fn: \
  None if arg == None else fn(arg)

divide = lambda a, b: \
  None if b == 0 else a / b
\end{Verbatim}
\end{quoting}
\end{multicols}\colskipb
Nevertheless, this is not the end of the story, because $1 - x$ might
be zero, and I want the function to return \verb|None| in such cases;
but considering that functions are only evaluated when applied to some
arguments, \verb|f3| can be transformed into the function above on the left,
with ancillary functions on the right.  Moreover, I did not tell you $x$ and
$y$ might be \verb|None| instead of numbers, but following the line of thought
above, we can further transform \verb|f4| into\footnote{Incidentally, this is
reminiscent of the continuation-passing style (\cf~Footnote~\ref{fn:cps}),
which is in fact not a coincidence\cupercite{troelskn2009}.}
\begin{quoting}
\begin{Verbatim}
f5 = lambda x, y: check(
  x, (lambda x: check(
    y, (lambda y: check(
      divide(y, 1 - x), (lambda y: check(
        x + y, (lambda x: x * x + y))))))))
\end{Verbatim}
\end{quoting}
Using this technique, we actually implemented exceptions in a purely functional
way\footnote{Conversely, we can also see that, as a matter of fact, exceptions
have a ground in type theory.  Besides, I find it necessary to note that
unlike Haskell, Lisp does not pursue purely functional programming, although
the functional style is certainly preferred.}; if you are familiar with Haskell,
you might realise this essentially using a monad.  A Haskell counterpart of
\verb|f5| can be written in the form below on the left, which is equivalent
to the form on the right using syntactic sugar hardcoded in Haskell\footnote%
{\texttt{return} in Haskell is just a function, and the two \texttt{return}s
on the right correspond exactly to the two \texttt{Just}s on the left,
which are needed due to type-theoretical requirements in Haskell.}.
\colskipa\begin{multicols}{2}
\begin{quoting}
\begin{Verbatim}
f5 = \x y ->
  (x >>= \x ->
    (y >>= \y ->
      (divide y (1 - x) >>= \y ->
        (Just (x + y) >>= \x ->
          Just (x * x + y)))))
\end{Verbatim}
\end{quoting}
\begin{quoting}
\begin{Verbatim}
f5 = \x y -> do
  x <- x
  y <- y
  y <- divide y (1 - x)
  x <- return (x + y)
  return (x * x + y)
\end{Verbatim}
\end{quoting}
\end{multicols}\colskipb\noindent%
So what is the role of Scheme and S-expressions here?  As was mentioned
just now, the syntactic sugar above is hardcoded in Haskell;
in Scheme, it can be implemented as a syntactic extension.

In the end of this section, I find it necessary to note that in the operating
system, \stress{directory trees resembles S-expressions} in that they are
essentially tree structures; the ``everything is a file'' principle of Unix
(\cf~Section~\ref{sec:plan9}) is a directly result of this idea, although
most important contributors to Unix probably did not realise the similar
potential of S-expressions: for example Daniel J.\ Bernstein noted, in the
context of security, that parsing and quoting of textual data significantly
complicate software systems\cupercite{djb:qmailsec}; instead, he extensively
used directory trees with very simple formats to configure his software, and
this practice is also adopted by some programmers with close relation
to his software, as can be seen in most daemontools-ish systems.

The complexity of parsing could have been greatly alleviated had we used
homoiconic formats, like S-expressions, when processing data with complicated
structures; the implications of homoiconicity in the context of Unix philosophy
will be further examined in the rest of this document, and here I instead
want to consider a little more about directory trees: given the similarity
between directory trees and S-expressions, why don't we simply replace most
configuration directories with files containing S-expressions?  In my opinion,
an S-expression file is indeed easier to work with than a directory tree is,
but it is also more rigid and monolithic; in comparison, with a directory
tree we can update parts of the configuration atomically, and can assign
different permissions to different parts.  Both of the latter are much
harder to achieve with a single file, which I believe shows the
advantage of directory trees as a configuration interface.

\section{The New Jersey and MIT / Stanford styles}\label{sec:wib}

In last section, we have seen the unrivaled power from homoiconcity, which
allows the macro system to easily manipulate ASTs in order to implement new
language features; furthermore, because the macro expansion is recursive,
macros in later parts of a program can access all language features defined
earlier.  When writing high-performance interpreters and optimising compilers,
homoiconicity enables us to not only write transformations on the processed
source code succinctly by using these \stress{structural macros}, but also
implement the transformations as multiple simple, clear and reliable passes,
similar to traditional Unix tools connected by pipes.  The prime example for
this \stress{multi-pass processing} scheme\footnote{\label{fn:slew}Noticing
the analogy between directory trees and S-expressions in Section~%
\ref{sec:homoiconic}, we can also implement multi-pass preprocessors for
configuration directories\cupercite{gitlab:slewman}; furthermore, from the
viewpoint of the Prototype Pattern, \texttt{fork()}/\texttt{exec()} and
Bernstein chainloading can also be regarded as similar designs.} is the
nanopass framework\cupercite{andersen2016}\footnote{Incidentally, the
name of one main author of nanopass, when abbreviated, becomes ``AWK''.},
which is used by Chez Scheme\cupercite{wiki:chez}, the prestigious Scheme
compiler that became open-source in 2016; Chez Scheme often produces
executables that are even faster than workalikes written in C, and yet it
has a codebase more than one order of magnitude smaller than that of GCC,
and self-compiles in seconds while GCC requires thousands of seconds.

As can already be seen above, in comparison with Lisp, C is in many aspects an
inferior language\cupercite{graham2002}, to put it bluntly, and the pros and
cons of both will be further examined in the next section; this is a result of
effects from multiple historical and methodological factors, and I summarise
them as the following: due to hardware limitations, C began as basically a
portable assembly language, reasonably simple to implement with acceptable
performance on computers at Bell Labs; due to the success of Unix, continued
efforts are made to create stronger optimising C compilers, so it is still one
of the most performant languages; and while it certainly has multiple weaknesses
in comparison with other languages, people usually either do not find the
weaknesses a major drawback, or simply use alternative languages.  In simple
words, C was chosen by Unix initially because the former was simple to implement
and fairly easy to use, and later mainly because of inertia; the first reason
highlights some kind of pragmatism present throughout the development of Unix,
and a well-known summary of it is Richard P.\ Gabriel's ``worse is better''%
\cupercite{dreamsongs:wib}\footnote{As you can see from the page, Richard
P.\ Gabriel had still not decided which style was better.  Additionally,
the complexity issue of interruptible system calls in Unix can be bypassed
simply by userspace wrappers that retry until the system calls return without
interruption; however, current standards curiously do not require such
wrappers, which is one reason why Daniel J.\ Bernstein created
his own interface set (\cf~Section~\ref{sec:devel}).}.

Gabriel compared Lisp with Unix and concluded that, basically, when simplicity
of the interface and simplicity of the implementation conflict\footnote{Richard
P.\ Gabriel later noted in \emph{Worse is Better is Worse} that people often
ignore this central premise, and instead just blindly aim to produce software
that only implements 50\% of the requirements.  Another sign of ignoring the
premise, I believe, is the objection to refactoring when attempting to fulfill
the complete requirements, with ``worse is better'' as an excuse\cupercite%
{chiusano2014}.}, Lisp would choose the former (``\stress{the right thing}'' or
the MIT / Stanford style), while Unix would choose the latter (``\stress{worse
is better}'' or the New Jersey style).  Apart from C, another main manifestation
of ``worse is better'' in Unix is the preference of textual interfaces by
traditional Unix tools\footnote{In addition to the plain text \vs~S-expressions
debate, there is also the text \vs~binary debate; in my opinion, from cases
like the use of CDB in qmail and s6-rc, we can see that binary files are not
always avoided, but only when there are simple ways to use text files without
compromising the requirements (\eg~performance and security).} (\cf~Section~%
\ref{sec:mcilroy}); and just like C, textual interfaces are often criticised in
the Lisp circle, because plain text is unstructured unlike homoiconic formats
(\eg~S-expressions).  However, from the perspective of minimising the system's
total complexity which covers both the interface and the implementation, I
believe that \stress{the choice between plain text and S-expressions is not
a black-or-white problem}: when processing data with simple structures, using
plain text is usually better because the interface is still easy enough to use;
in contrast, when processing data with complicated structures, the decrease in
interfacial complexity of using S-expressions often outweighs the increase
in implementational complexity.  When dealing with languages, extensive
processing of data with deeply nested and often self-referencing
structures are necessary, so S-expressions are undoubtedly preferable.

Another common criticism against Unix from the Lisp circle is that the latter
relentlessly exposes low-level details to the user, and that instead these
details should be hidden from the user by means of encapsulation\footnote{From
this criticism, we can also understand why Lisp focuses more on the interfacial
complexity.}; however, a common problem with encapsulation is \stress{leaky
abstractions}, where customisation often has to be achieved by working around
the encapsulation.  systemd and Linux distributions from Red Hat, examined
in the first part of this document, are the more severe examples; actually
I believe there are few abstractions that are satisfactorily airtight,
and among them are certain well-designed programming languages as well
as kernels of Unix-like operating systems.  Even when we distinguish
between ``normal'' and ``advanced'' users, providing encapsulation
to the former without interfering with the latter's convenience
in customising and debugging does not seem like a solved problem.

\section{Benefits of reconciling Unix and Lisp}\label{sec:benefits}

As noted in last section, C is an inferior language to Lisp, most importantly
due to its lack of expressiveness: for example, had it used a macro system (like
that in Dale\cupercite{github:dale}) that could modify ASTs, then features like
object-oriented programming would be implementable using macros, and C++ would
perhaps have been unnecessary.  Another prominent weakness of C, widely noted
both inside and outside the Lisp circle, is its memory unsafety which results
in buffer overflows, null-pointer dereferences \etc; alternative C libraries
like skalibs (\cf~Section~\ref{sec:devel}) surely help to reduce this kind
of bugs, but are far from able to reducing them to the minimum, and
I will mention a possible strategy for this in the next section.

However, to be fair, Lisp also has its weaknesses, among which a best-known is
the performance cost of dynamic type checking, especially in scenarios like
numerical computation; automatic type inference, even in its advanced form in
\eg~Chez Scheme, is not a universal solution to this problem.  Another main
problem is the necessity of garbage collection in Lisp implementations, which
makes compiled executables larger in size and often unsuitable for use in
real-time environments; furthermore, advanced requirements for low-level
development, for instance cryptography (\cf~NaCl and BearSSL) which often
requires both constant-time execution and high performance (often involving
assembly), seem largely ignored in the Lisp circle. To summarise them up,
from what I know, I feel that while Lisp is good at implementing complicated
application requirements through abstractions, it is less involved
in low-level development, where it has big potentials in fact.

Something interesting lies in the observations above -- the pros and cons
of Lisp and C seem to be complementary; this naturally leads to the guess
that if we somehow manage to design \stress{a minimalist homoiconic language
that combines their strengths while avoiding their weaknesses}, it would be,
in Tony Hoare's words\cupercite{hoare1981}\footnote{A most important lesson we
have to learn from the speech, as Tony Hoare put it, is that simplicity must
be taken as the first priority when pursuing an ultimate language.},
\begin{quoting}
	\emph{[...]} a language to end all languages, designed to meet the needs
	of all computer applications, both commercial and scientific, \emph{[...]}
\end{quoting}
But is that achievable?  If yes, how to do it, and what would we get?
If no, would the attempt be worthwhile?  My guess for the first question
is ``yes''; the second and the fourth will be respectively explored
in the next section and Section~\ref{sec:toe}; in the rest of
this section, I will give some examples for the third question.

As was mentioned in Section~\ref{sec:exec}, Laurent Bercot implemented the
unit operations in chainloading as a set of command-line tools; in order to
implement unit operations involving two commands (\eg~loops and pipelines),
he designed a special quoting syntax for command line arguments, and wrapped
it up as ``blocks''\cupercite{ska:elblocks} in his language called execline%
\footnote{Its design is similar to the initial shell in Unix, the Thompson
shell\cupercite{wiki:thompson}, which was substituted by the Bourne shell in
Unix v7 due to its inadequacy for more complex programming.}.  Nevertheless,
chainloading does not necessary imply \verb|exec()|, and instead can be done in
a shell-like language\cupercite{vector2016a} (\cf~shell commands like \verb|cd|
and \verb|umask|); based on the model of scsh\cupercite{scsh:home}, the execline
language can be replaced with scsh-like Scheme, with chainloaders replaced by
scsh-like Scheme programs analogous to the following shell reimplementation of
the \verb|cd| chainloader provided by execline\cupercite{vector2018c}:
\begin{quoting}
\begin{Verbatim}
#!/bin/rc -e
cd $1; shift; exec $*
\end{Verbatim}
\end{quoting}

Following this line of thought, it is not unimaginable to replace traditional
Unix tools with Scheme programs as well; considering the expressiveness of
Scheme and the existence of elegant Scheme compilers like Chez Scheme, this
would surely help to reduce the total complexity of the system.  So now we
see that by migrating to Lisp for system programming, even these simple
tools can be further simplified; and as was noted before, Lisp is probably
even more powerful for application programming, so I find it undoubted
that \stress{by embracing homoiconicity, we will be able to construct
systems that are more Unix-ish than was possible before}.

Now that we have seen the power of homoiconicity in system programming, it is
appropriate to revisit the Trusting Trust issue in Section~\ref{sec:security};
at that time I noted that one method against Trusting Trust is to construct the
compiler from the machine code in multiple steps, and it should be self-evident
that the auditability of this procedure directly depends on the latter's total
complexity.  It was also noted in Section~\ref{sec:lisp} that Steve Russell
implemented the first Lisp interpreter in assembly by hand, which in combination
with the elegance of Chez Scheme tells us that \stress{homoiconicity will
dramatically reduce the total complexity of bootstrapping from machine code},
and therefore strengthen our defence against Trusting Trust\footnote{Another
potential problem is hardware backdoors, and I think a way to deter them is to
leave adequate room for variation in every step in the bootstrapping procedure,
because the semantics of low-level code is difficult to analyse (which results
in the importance of open source), so any attempt at mechanically injecting
malicious code may trigger false positives, which may expose the backdoor.}.

\section{Lisp / C reconciliation: how to?}\label{sec:howto}

As was emphasised in Section~\ref{sec:devel}, we should analyse specific issues
specifically, and while the disappearance of resource limitations does not imply
simplicity was no longer important, it surely motivates us to rethink previous
compromises made due to these limitations.  We are well past the era where
garbage collection was prohibitively expensive for regular computers, and now
for devices with severely limited resource we generally prefer cross-compilation
over native compilation.  Given this observation while considering the issue
of big executables and real-time requirements in last section, the proposed
language (I give it the codename ``Nil'', for uNIx plus Lisp) should be able
to use GC for its interpreter and compiler, but \stress{executables produced
by the compiler should avoid using GC whenever reasonable}\footnote{However,
since I proposed in Section~\ref{sec:benefits} to substitute the shell with
Nil (in the guise of Scheme), there does exist the natural question of how
a Nil interpreter should exist in those special environments.  I do not yet
have a good answer for it; or we could perhaps use an interpreter stripped
down specifically for shell emulation?}, and the programmer should be
aware of how to avoid GC when implementing special requirements.

In last section, we revisited the issue of bootstrapping, and I hinted about
a procedure where a minimal interpreter is somehow built, and then in a later
step a production-ready compiler is produced; a first point to note is that
this deviates from today's common practice of using a compiler to build an
interpreter, and instead constructs an interpreter first.  However, as can be
guessed, Nil interpreters would probably be easier to write in assembly than
Nil compilers are, just as with Lisp which Nil is modeled upon, so from the
viewpoint of complexity it would be better to generate the compiler from a
primitive interpreter\footnote{Probably in multiple steps, involving more and
more powerful compilers, interpreters and even assemblers, and the assemblers
would probably somehow resemble the proposed Nil-powered assembly; to facilitate
auditing, a smaller number of intermediate steps should probably be preferred,
as long as the total size of the codebase involved is close to the minimum.
Besides, on a more theoretical level, I find the notion of Futamura projections
(essentially \stress{partial evaluation})\cupercite{wiki:parteval} highly
interesting.}.  Similarly, noticing the expressiveness of Nil, it would
be natural to use a self-bootstrappable Nil compiler as the base
compiler in production systems, from which other compilers (if
still necessary, and perhaps including one for C) are produced.

Hitherto in this section, we have been examining the Lisp-like aspects of Nil,
but what about the C-like aspects?  I consider Dale, mentioned in last section,
a possible prototype for a C-like layer, upon which a Lisp-like layer would
be implemented using macros, and many other languages (\eg~shell and Makefile)
would be emulated in turn on the Lisp-like layer, accesible to the user via
something like \hologo{LaTeX}'s \stress{macro packages}.  Another approach I
have thought of, which I give the codename Nil-powered assembly, is based on the
notion of a portable assembly\footnote{Since the C-like layer would essentially
be a machine code generator, similar generators like Chez Scheme's built-in
assembler and Daniel J.\ Bernstein's qhasm might be highly instructional
references.} (\cf~Section~\ref{sec:wib}): the C-like layer would be implemented
based on the Lisp-like layer as most other languages would be, and it
would be a markup-like ``reverse-macro'' system which provides primitive
procedures that when called emit assembly instructions to the output.

In addition, as was noted in last section, Nil needs a type system strong than
that of Lisp, for which I find Yin Wang's design base on \stress{contracts}%
\cupercite{wangyin2013a, wangyin2013b} a very thought-provoking reference; it is
worth noting that this design helps to boost memory safety of the C-like layer,
and even facilitates the formal verification of written programs (\cf~%
Footnote~\ref{fn:formal}).  The foreign function interface between the Lisp-like
and C-like layers will also be an issue, which I am not yet knowledgeable enough
to comment on.  Finally, before ending this section, I need to stress the
non-technical issue of inertia, which troubles Scheme (\cf~Footnote~%
\ref{fn:r6rs}), Plan~9 (\cf~Footnote~\ref{fn:plan9}) as well as qmail and
other excellent (or not) software projects: apart from the technical challenges
Nil will inevitably encounter, it will also have to attract some attention
from both academia and industry to gain enough momentum (there is surely the
possibility that the latter simply does not care, but see the next section); and
even if Nil might have some success, getting rid of historical burdens while
adapting to new requirements would not be easy for it, as is for other projects.

\section{Toward a ``theory of everything''}\label{sec:toe}

As was noted in last section and Section~\ref{sec:benefits}, Nil might be a
language that is impossible even in the theoretical sense, or might not attract
adequate attention from the industry, so would the exploration on Nil still be
worthwhile in such cases?  In this final section, I will give my answer
to this question based on the analogy I made in \parencite{vector2018c}:
\begin{quoting}
	I guess reconciling Lisp and Unix would be much easier than
	reconciling quantum mechanics and general relativity; and
	it would be, in a perhaps exaggerated sense, as meaningful.
\end{quoting}

Theoretical physicists and high-energy physicists have long pursued a theory
of everything that can reconcile quantum field theory and general relativity,
because the theory would provide a unified basis for all low-level physical
phenomena.  To my knowledge, there does not seem to be a firm guarantee such
a theory can be successfully achieved by humans: for example, the best-known
candidate, string theory, still appears far from testable, even though it does
agree with observed facts.  On the other hand, there is not a proof that such
a theory is unachievable, either; and apart from the theoretical promises,
research on it has always been giving rise to exciting advancement in diverse
fields ranging from pure mathematics (which is profoundly connected with
theoretical physics) to civil engineering (involved in construction of
large facilities for physical experiments).  For these two reasons,
a ToE is still the lifelong pursuit for many physicists.

Similarly, although the efforts at Lisp / C reconciliation is not guaranteed
to really result in an ultimate language, it would surely be rewarding to
explore \stress{the best way the two languages can collaborate}, so that the
system's complexity is minimised and the programmer's productivity is maximised:
as has already been seen in Section~\ref{sec:benefits}, the reconciliation, even
in a very primitive form, can already contribute greatly to the reduction in the
complexity of Unix systems; I believe that further simplification of these
systems will attract more people to explore the fundamental aspects of computer
systems, just like how the Raspberry Pi reignited the enthusiasm for low-level
development.  And even if our efforts might not result in production-ready
Nil-based systems finally, I am completely convinced that insightful
communication between the Unix and Lisp circles will, as was explained in
Section~\ref{sec:worklife}, definitely result in a lot of \stress{byproducts
that are both interesting and meaningful}, therefore I conclude this document
with two quotations respectively from Rob Pike\cupercite{pike2000}%
\footnote{The description is, ironically, applicable to himself as well
regarding programming languages.} and Hal Abelson\cupercite{abelson2008}:
\begin{quoting}
	Narrowness of experience leads to narrowness of imagination.
\end{quoting}
\colskipc
\begin{quoting}
	If you \emph{[carefully study the example interpreters in this book]},
	you will change your view of your programming, and your view of
	yourself as a programmer.  You'll come to see yourself as a designer
	of languages rather than only a user of languages, as a person
	who chooses the rules by which languages are put together, rather
	than only a follower of rules that other people have chosen.
\end{quoting}

