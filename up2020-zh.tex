\documentclass[UTF8]{ctexart}
\usepackage{setspace, footmisc, hyperref, enumitem, fancyvrb, tikz}
\usepackage{mdframed, multicol, amsmath, bm, forest, hologo, manfnt}
\usepackage[a4paper, margin = 2.5cm]{geometry}
\usepackage[titles]{tocloft}
\usepackage[
	backend = biber, style = caspervector, utf8, giveninits = true
]{biblatex}
\usetikzlibrary{calc, math, positioning}
\addbibresource{up2020.bib}
\newcommand*{\docversion}{v0.1.3}

\pagestyle{plain}
\setlength{\hfuzz}{3pt}
\setlength{\cftsecnumwidth}{2em}
\setlist{nolistsep, leftmargin = 3em}
\renewcommand{\thesection}{\ifnum\value{section}<10 0\fi\arabic{section}}
\renewcommand{\footnotelayout}{\ }
\renewcommand*{\bibfont}{\small}
\appto{\bibsetup}{\setlength{\hfuzz}{6pt}}
\newcommand*{\cmc}{black!66}
\newcommand*{\cmm}{\color{\cmc}\rmfamily\itshape}
\newcommand*{\cm}[1]{\cmm{#1}}
\tikzset{x = 1pt, y = 1pt}
\newcommand*{\tikzbase}{\tikzmath{
	\unity = \baselineskip; \unitx = 1.02 * \unity;
	\distx = 0.9em; \disty = 0.64em;
}}
\newcommand*{\tikzproc}[3]{\node at (#1 * \unitx - \distx, -#2 * \unity)
	[anchor = west] {\texttt{#3}}}
\newcommand*{\tikzcomment}[3]{\node at (#1, -#2 * \unity)
	[anchor = west] {\cm{#3}}}
\newcommand*{\tikzvline}[2]{
	\draw (#1 * \unitx, {-\disty - (#2 - 1) * \unity})
		-- (#1 * \unitx, -#2 * \unity)
}
\newcommand*{\tikzhline}[2]{
	\draw (#1 * \unitx, -#2 * \unity)
		-- ({(#1 + 1) * \unitx - \distx}, -#2 * \unity)
}

\newcommand{\specialsec}[1]{%
	\section*{#1}\addcontentsline{toc}{section}{#1}%
	\markboth{#1}{}\phantomsection%
}
\newcommand{\newpartx}[1]{%
	\addtocontents{toc}{\protect\vspace{#1\protect\baselineskip}}
	\clearpage%
}
\newcommand{\newpart}{\newpartx{1}}
\newcommand*{\stress}[1]{\textbf{#1}}
\newcommand*{\cupercite}[1]{\supercite{#1}\mbox{}}
\newcommand*{\colskipc}{\vspace{-0.25\baselineskip}}
\newcommand*{\bs}{\char92}
\RecustomVerbatimEnvironment{Verbatim}{Verbatim}%
	{baselinestretch = 1.2, formatcom = {\ifXeTeX\xeCJKVerbAddon\fi}}
\mdfdefinestyle{leftbar}{
	rightline = false, topline = false, bottomline = false,
	leftmargin = 2.45em, innerleftmargin = 0.65em,
	rightmargin = 3em, innerrightmargin = 0pt,
	innertopmargin = 0.36em, innerbottommargin = 0.3em,
	skipabove = {0.36\baselineskip}, skipbelow = {0.1\baselineskip}
}
\newmdenv[style = leftbar]{quoting}
\newmdenv[style = leftbar, rightmargin = 0pt]{wquoting}

\makeatletter
\newbibmacro*{eid+url+urldate}{%
	\printfield{eid}\setunit*{\bbx@cecomma}%
	\printfield{url}\setunit*{\bbx@cecomma}%
	\iffieldundef{urlyear}{}{%
		\printtext{\bbx@cetext{\bbx@cnretr}{accessed on}\addspace}%
		\iffieldundef{verba}{\printurldate}%
			{\href{\thefield{verba}}{\printurldate}}%
	}%
}
\makeatother


\newcommand*{\rnrs}[1]{R$^{#1}$RS}
\newcommand*{\colskipa}{\vspace{-0.55\baselineskip}}
\newcommand*{\colskipb}{\vspace{-0.45\baselineskip}}

\begin{document}
\title{\textbf{Unix 哲学 2020}}
\author{Casper Ti.\ Vector \texttt{<CasperVector@gmail.com>}}
\date{2019 年 8--9 月(\docversion)}
\maketitle
\vspace{\baselineskip}
\tableofcontents

\newpartx{0.5}
\specialsec{序}

我从 2008 年末开始学习 Linux,并从 2009 年初起使用 Linux,从此至今 Linux 一直是
我的主操作系统。在使用 Linux 以及在类 Unix 操作系统圈内活动的过程中,我逐渐被
“Unix 哲学”这一理念所吸引,而这种观念也给了我巨大的收益。(关于本文档来源的
更多细节,可以参考\hyperref[sec:afterword]{跋}。)本文档是我关于 Unix 哲学
这一主题相关思考的一个总结,主要分为下文所概括的三个部分。

在第一部分中,我们将首先讨论“\stress{Unix 哲学的本质是什么}?”这一问题,其中
我将论证其本质是在几乎满足需求的前提下最小化系统的认知复杂度。进年来,不少
人认为来源于 1970 年代计算资源限制的 Unix 哲学已不再适用于当今的软件开发,
那么 \stress{Unix 哲学还有没有意义},它在 2020 年代是否还会有意义?我的回答是
“有,而且比它诞生之时更加重要”。在本部分的结尾,我将就\stress{怎样在实际开发中
贯彻 Unix 哲学}并产生实际的收益。

在第二部分中,我们将把关注点移到编程领域之外,来探索 \stress{Unix 哲学在科学、
技术和社会中的适用性,以及这一适用性的限度}。在关于科学技术的讨论之后,我将考虑
哲学、文学和艺术中的极简主义,并试图寻找极简主义的认知本源,而这一本源将用于解释
为何社会在很多方面并不像 Unix 那样工作。在本部分的结尾,我们将讨论\stress{个人,
特别是并不出众的人,可以怎样应用极简主义}来提升自己在日常工作中的效率。

在第三部分中,我们将回到编程的话题,但这次我们先暂时离题去考虑 Lisp,特别是
Scheme,因为这个圈子的人也很鼓励极简主义。为了理解 Lisp 的极简主义,我们将
考察同像性的概念,我认为这一概念和复杂度具有同等的重要性。之后我将尝试
回答“\stress{Unix 和 Lisp 之间分歧的根源在哪里}?”这一问题,并最终探索%
\stress{一种吸收 Lisp 和 C 双方优点的极简主义语言框架}的可行性和好处,
而这种框架也将避免上述两种语言的缺点。

在实际开始正文之前,有必要特别指出本文档应以一种批判性的眼光来阅读,其原因
有两点。第一,这里写的只是我对 Unix 哲学的个人看法,而后者又只是对世界进行建模
的许多方法之一。第二,我对哲学、认知科学、程序语言研究等陌生领域仍然知之甚少,
因此关于这些话题的内容具有很强的试探性质,其中很多论点可能是业余甚至荒谬的。
在本文档中我尽量避免出现错误,如果你发现任何错误也欢迎告诉我。

本文档是 2018 年 11 月 10 日我在北京大学 Linux 俱乐部所作技术报告《Unix
哲学和当今》\nolinebreak\cupercite{casper2018}文字整理版的扩充;我视其为
我在开源社区 10 年经验的结晶,并将其献给 Unix 的 50 周年\cupercite{salus2005}。
作上述报告以及编写本文档对我是愉快的经历,而我也希望本文档能给你留下一些
有意思的点滴,无论你是初来开源界的新人,还是几乎 3 倍于我年龄的专家。

\newpart
\section{Unix 历史简介}\label{sec:intro}

CTSS\cupercite{wiki:ctss}(最早发布于 1961 年)被广泛认为是历史上第一个分时
操作系统;它也是一个相当成功的系统,而其成功催生了更加有野心得多的 Multics%
\cupercite{wiki:multics} 项目(其开发开始于 1964 年);然而尽管有 MIT、GE 和
Bell 实验室的共同努力,Multics 直到 1969 年才产生首个可商用的系统\cupercite%
{multicians:history},原因在于其就当时的人力和计算资源而言过于复杂\footnote{%
\label{fn:multics}事实上,Multics 的硬件需求并不强于今日的低端家用路由器,而相比
之下现在的 Linux 只能在经过高度裁剪之后才能运行在这样的硬件上。Multicians 网站上
有一个页面\cupercite{multicians:myths}对关于 Multics 的常见误解进行澄清。}。Bell
实验室在 1969 年退出了 Multics 项目,而之前在此项目上工作的 Ken Thompson 和
Dennis Ritchie 转而开发另一个操作系统来满足自己的需求,这个系统后来得名“Unix”%
\cupercite{wiki:unixhist}。为了让新系统在当时 Bell 实验室中一台(多少)闲置的
PDP-7 小型机上可用且可维护,Thompson 对 Multics 的设计进行了大规模简化,只
保留了其中如层状文件系统和 shell 等等的少量关键元素\cupercite{seibel2009}。

1970 年代见证了 Unix 的发展传播(详见 \parencite{wiki:unixhist}),我认为其中最
重要的历史事件是 1976 年《Lions' Commentary on Unix v6》\cupercite{wiki:lions}%
\footnote{Unix v6 的一个现代移植版本见 \parencite{wiki:xv6}。}一书的发表,因
其极大地促进了 Unix 在各大学中的传播。1980 年代初期出现了来自多家供应商的商业
Unix 版本,但 AT\&T(当时拥有 Bell 实验室)却因反托拉斯的限制而不能商业化 Unix;
情况在 1983 年发生了变化,此时 Bell 系统被拆分,于是 AT\&T 迅速把 Unix 商业化,
并限制其源代码的传播。源代码交流上的约束加剧了已经出现的 Unix 碎片化,造成了
我们现在所称的“Unix 战争”\cupercite{raymond2003a},而这一战争以及 Unix 圈对
80x86 微型机潜力的忽视导致了 1990 年代 Unix 在流行度上令人惋惜的衰退。

1991 年,Linus Torvalds 开始开发他自己的操作系统内核,而后者成为了 Linux;GNU
项目(开始于 1985 年)所提供用户空间工具和 Linux 内核的结合达成了提供一个自由、
开源、低成本的自托管类 Unix 系统的目标\footnote{386BSD 项目也达成了这一目标,但
其首次发布是在 1992 年;此外,当时的一场诉讼以及社区内讧\cupercite{wiki:386bsd}%
分散了 BSD 人的精力,而现在或许可以说从那时开始 BSD 再也没能在流行度上赶超
Linux。},从而催生了 GNU/Linux 这一生态系统,而后者可能是开源运动中最重要的
阵地。当今最流行的类 Unix 系统毫无疑问是 Linux 和 BSD,而 Solaris、QNX
等等商业系统只有较小的市场占有率;一个不流行但很重要的系统是 Plan~9 from
Bell Labs(最早发布于 1992 年),而我将在第 \ref{sec:plan9} 节中对其进行介绍。

在结束本节(截止目前主要是非技术的内容)之前,我希望强调 Thompson 和
Ritchie 自己认为“影响了 Unix 设计”的三点技术考虑\cupercite{ritchie1974},
这三点都将在后面的章节中涉及:
\begin{itemize}
\item \stress{对程序员友好}:Unix 被设计为提升作为程序员的用户工作效率的系统;
	另一方面,我们也将在第 \ref{sec:user} 节中讨论 Unix 方法论对普通用户的价值。
\item \stress{简洁}:1970 年前后 Bell 实验室中机器的硬件限制导致了 Unix 对
	经济和优雅的追求;这样的限制早已过时,那么追求简洁只有审美意义了吗?
	我们将在第 \ref{sec:quality}--\ref{sec:foss} 节讨论这一问题。
\item \stress{自托管}:即使最早的 Unix 系统也能独立维护,而不
	依赖运行其它系统的机器;这要求系统能自举,而后者的意义将在第
	\ref{sec:security}、\ref{sec:benefits}--\ref{sec:howto} 节中讨论。
\end{itemize}

\section{Shell 编程一瞥}\label{sec:shell}

上节提到,shell 是 Unix 从 Multics 中借鉴的少数设计元素之一;事实上,
作为除图形界面外普通用户和系统交互的主要渠道\cupercite{ritchie1974},shell
也是 Unix 中最体现其设计思想的组件之一。作为例子,我们可以看经典的词频排序问题%
\cupercite{robbins2005}——编写程序输出指定文本文件中出现最多的 $n$ 个单词及其
频数,这一问题吸引了 Donald Knuth、Doug McIlroy 和 David Hanson 前来给出解答。
Knuth 和 Hanson 的程序分别采用 Pascal 或 C 从头编写,两者的编写和调试均
花费数小时;McIlroy 的程序是一个 shell 脚本,其编写只需一两分钟,
且第一次运行即通过,这一脚本稍加修改后如下:
\begin{quoting}
\begin{Verbatim}
#!/bin/sh
tr -cs A-Za-z\' '\n' | tr A-Z a-z |
sort | uniq -c |
sort -k1,1nr -k2 | sed "${1:-25}"q
\end{Verbatim}
\end{quoting}

其首行告诉 Unix 这是一个由 \verb|/bin/sh| 解释执行的程序,
剩下各行命令由\stress{管道}“\verb@|@”分为 6 步:
\begin{itemize}
\item 第 1 步的 \verb|tr| 命令将除大小写英文字母和“\verb|'|”字符外的
	所有字符(\verb|-c| 选项指定取补)替换为换行符 \verb|\n|,
	其中 \verb|-s| 选项指定将连续多个换行符替换为单个。
\item 第 2 步命令把所有大写字母替换为相应的小写字母;
	经过这一步,输入的文本被变换为每行一个全小写单词的形式。
\item 第 3 步的 \verb|sort| 命令将所有行按字典顺序排序,
	于是相同的单词必然在相邻的行输出。
\item 第 4 步的 \verb|uniq| 命令把连续多个相同的行替换为单个,在加上
	\verb|-c| 选项之后会在行首添加该行重复的次数(即词频)。
\item 第 5 步的 \verb|sort| 命令将各行根据每行第 1 字段(即上一步添加
	的频数)的数值从大到小排序(\verb|-k1,1nr| 选项,其中 \verb|n|
	默认从小到大,\verb|r| 则指定按相反规则),在频数相同时根据第
	2 字段(即单词本身)按字典顺序排序(\verb|-k2| 选项)。
\item 第 6 步的 \verb|sed| 命令只打印其输入中最靠前的若干行,
	而行数由脚本执行时的第 1 个参数决定,该参数为空时默认行数为 25。
\end{itemize}
除了便于编写、调试之外,这个脚本也具有很强的可维护性(也包含可定制性),
因为其各个步骤对输入的要求以及处理规则十分简洁清晰,我们可以轻松地替换
其中一些步骤的具体实现:例如上述的分词判据显然很粗糙,其中并未考虑
单词形态变化(如“look”“looking”和“looked”看成同一单词)等问题;如果
要在上述脚本中实现这样的需求,我们只须把头两步替换为其它的分词程序
(大概须要单独编写),并设法让它按和原来相同的接口进行输入/输出。

和许多 Unix 工具(例如上文中用到的几个工具)类似,该脚本从\stress{标准输入}%
(默认为键盘)读取输入,并将结果写到\stress{标准输出}(默认为当前终端)%
\footnote{C 语言中的 \texttt{stdio.h} 指的就是标准输入/输出。};利用 Unix 的%
\stress{输入/输出重定向}机制可以实现从/到指定文件的输入/输出,例如在 shell
中运行以下命令(假定上述脚本名为 \verb|wf.sh| 且已赋予执行权限)
\begin{quoting}
\begin{Verbatim}
/path/to/wf.sh 10 < input.txt > output.txt
\end{Verbatim}
\end{quoting}
即可将 \verb|input.txt| 中出现最多的 10 个单词及其频数输出到 \verb|output.txt|。
显然,管道也是一种重定向机制,它把左边命令的输出重定向到右边命令的输入;换一个
角度,如果把被管道连接的各个命令看成一个个过滤器,那么每个过滤器所做的就是相对
简单的操作,而像上述脚本那样的编程方式本质上就是将复杂的文本处理任务分解为通过
管道环环相扣的多个简单过滤步骤,并用相对现成的工具实现相应的过滤器。从上面的
例子,我们已经可以初步感受到 Unix 工具组合起来所能达成的强大威力;然而,包括
Windows 在内的其它一些系统中往往也有和 Unix 中输入/输出重定向和管道等等
相似的机制,那么为什么我们在这些系统中不常见到类似的用法呢?请看下节。

\section{软件工程中的内聚和耦合}\label{sec:coupling}

内聚和耦合是软件工程中极为重要的概念,这里我们先来了解什么是耦合:考虑
\parencite{litt2014a} 图中两种极端情形下模块之间的交互关系(可以是通过文本或
二进制字节流的通信,通过数据包或其它载体的消息传递,子程序之间的调用关系等等),
假如系统出现故障须要调试时,两者的调试难度分别如何?假如系统需求有变须要调整时,
两者的维护难度分别如何?我想答案应该非常明显。同样是由 16 个模块构成的系统,
模块之间交互关系的复杂程度决定了在调试、维护难度上的天壤之别,而\stress{耦合度}
正好可以理解为对这种交互关系复杂程度的度量;显然,我们希望系统中各模块尽量
低耦合,而上节的脚本易调试、易维护也正是其内部各命令之间低耦合的结果。

事实上,低耦合是传统 Unix 工具的普遍特点,那么为什么这些工具能实现低耦合?
从上节的脚本中,我们已经可以窥见其中端倪:它们不仅有着明确的输入/输出接口,而且
从输入到输出有着清晰的处理规则,或者说它们的行为有着明确的目标;从系统和模块的
角度来看,看作模块的各个 Unix 工具往往分别实现不同的单元操作,如字符转换
(\verb|tr|)、排序(\verb|sort|)、去重和计数(\verb|uniq|)等等。Unix 工具在
实现时内部往往也要划分子模块,但其子模块之间的耦合即使在最优设计下也不得不远高于
工具之间的耦合;我认为\stress{内聚度}正是对这种模块之间本质性耦合的度量,而按照
高内聚的原则来划分模块会自然地迫使各模块进行明确的分工,从而降低系统的耦合。

我们已经看到,高内聚、低耦合是我们希望软件系统能具有的属性。你可能会问,许多
Windows 程序也低耦合(例如记事本和画图程序互不依赖)且在一定程度上高内聚(例如
记事本用于编辑文本,画图程序用于画图),那么我们为什么不经常把它们像 Unix 工具
那样组合起来用呢?答案其实很显然——因为它们没有被设计成可组合;说得更明确一点,
就是它们不能使用某种像管道一样的简洁通用的接口来协作,因此难以在自动化任务中方便
地重用。相比之下,我们在上节中看到的 Unix 强大之处正是在于其强调用户所使用工具在
自动化任务中的可重用性,而这也导致高内聚和低耦合的原则在传统 Unix 工具中几乎体现
到了极致\cupercite{salus1994}。总结起来,对内聚和耦合的要求必须放在\stress{协作
和重用}的背景下考虑,而追求协作和重用也自然地促进高内聚、低耦合的设计。

截止目前,我们看到的例子都相对理想化或简单化,这里我再举两个更加实际且和近年
热点话题十分相关的例子。Unix 系统在启动时首先由内核创建第一个进程,并由该进程
创建其它一些进程,这些进程共同管理系统服务;因为其中第一个进程在系统初始化中的
重要地位,这个进程往往被称为“\stress{init}”\cupercite{jdebp2015}。systemd 是
目前最流行的 init 系统,其 init 功能十分复杂、机制缺乏文档描述;此外,除了名为
\verb|systemd| 的 init 程序以及相关辅助程序之外,systemd 还包含了许多其它非
init 的模块,而所有这些模块之间有着复杂且缺乏文档说明的交互关系(其一种比较
夸张的描绘可见 \parencite{litt2014b})。显然 systemd 是低内聚、高耦合的,
但事实上这种低内聚、高耦合并非必需,因为 daemontools 式的设计(本文档
以 s6 为其代表)比 systemd 简洁得多,但功能却不弱于 systemd。

s6 的 init 程序 \verb|s6-svscan| 扫描指定目录(“scan directory”,如
\verb|/service|)下的子目录,并对每个子目录(“service directory”,如
\verb|/service/kmsg|)运行一个 \verb|s6-supervise| 进程,后者通过运行 service
directory 中名为 \verb|run| 的可执行文件(如 \verb|/service/kmsg/run|)运行相应
的系统服务;用户可以使用 s6 提供的命令行工具 \verb|s6-svc|/\verb|s6-svscanctl|
来和 \verb|s6-supervise|/\verb|s6-svscan| 交互,而且可以利用 service directory
和 scan directory 中一些辅助性的文件调整 \verb|s6-supervise| 和 \verb|s6-svscan|
的行为\footnote{这样的配置方式可能显得不太直观,而第 \ref{sec:homoiconic} 节
和脚注 \ref{fn:slew} 将解释这样设计的理由和好处。}。s6 只管理长期运行的服务,
短期运行的 init 脚本由 s6-rc 负责,后者也通过 s6 提供的工具跟踪服务的启动
状况以实现对服务间依赖的管理。上文提到这样的设计在功能上不弱于 systemd,
我在这里举一个例子(一些更深入的例子见第 \ref{sec:exec} 节):systemd
支持服务模版,例如定义名为 \verb|getty@| 的模版后,\verb|getty@tty1| 服务
将会在 \verb|tty1| 上运行 getty 程序;在 s6/s6-rc 中,类似的功能可以通过在
\verb|run| 脚本中调用一个 5 行的库脚本\cupercite{gitlab:srvrc}来实现。

\section{Do one thing and do it well}\label{sec:mcilroy}

Unix 式的设计原则常被称为“\stress{Unix 哲学}”,其
最流行的表述无疑源自 Doug McIlroy\cupercite{salus1994}:
\begin{quoting}
	This is the Unix philosophy:  Write programs that do one thing and
	do it well.  Write programs to work together.  Write programs to
	handle text streams, because that is a universal interface.
\end{quoting}

结合上节的讨论,我们不难注意到 McIlroy 所说的第 1 点强调的是高内聚、低耦合%
\footnote{顺便提到,这也说明了高内聚、低耦合的要求事实上并不是面向对象编程
独有的;事实上,有人认为\cupercite{chenhao2013}面向对象编程中所有的设计模式都
可以在 Unix 中找到对应(一个例子见第 \ref{sec:exec} 节)。},第 2 点强调的是
程序的协作和重用,而第 3 点似乎稍显陌生:在第 \ref{sec:shell} 节中,我们的确
看到了文本处理工具相结合所产生的威力,但断言文本是一种通用的接口的深层理由
是什么?我认为这可以用\stress{人机接口}对人和对计算机的友好程度来解释(第
\ref{sec:cognitive} 节将再次涉及这一问题),即文本流是介于二进制数据和图形
界面之间的一种折衷选择:二进制数据方便计算机处理但很难被人理解,而且不同
处理器对其处理方式的微妙区别还带来了以大小端问题为代表的编码可移植性问题;
图形界面最方便人理解,但编写起来明显比文本接口复杂,且至今仍然不便协作%
\footnote{有必要指出,我并不排斥图形界面,而只是认为其设计有必要考虑自动化
的需求;而据我所知,图形界面的自动化至今仍然是一个不简单的课题。我目前
认为像 AutoCAD 那样在图形界面之外还有一个命令行界面,而操作图形界面时
在命令行上自动出现等价命令的设计应该是一种很好的思路。};文本流
既方便计算机处理也比较方便人理解,其虽然也涉及字符编码问题,
但后者总体上仍比二进制信息可能遇到的编码问题简单很多。

McIlroy 的表述并非没有争议,其中以文本流作为通信格式是不是最佳选择是最主要
的争议焦点,我们将在第 \ref{sec:wib} 节中进一步讨论这一问题;此外,这一表述
的确几乎涵盖了截止目前我们看到的让 Unix 强大的原因,但我认为其并不能完全代表
Unix 哲学。有必要指出,管道的出现直接导致了 Unix 先驱对命令行程序协作和重用的
追求\cupercite{salus1994},而 McIlroy 是 Unix 管道的发明者,因此他的总结明显
是立足于 shell 编程的。Shell 编程固然重要,但它远非 Unix 的全部:在接下来
两节中,我们将看到 shell 编程之外的体现 Unix 哲学的一些重要例子,它们
并不能被 McIlroy 的经典表述概括;然后在第 \ref{sec:complex}
节中,我将提出我所认为的 Unix 哲学的本质。

\section{\texttt{fork()} 和 \texttt{exec()}}\label{sec:exec}

进程是操作系统中最重要的概念之一,因此用于管理进程的操作系统接口具有一定的
重要性;每个进程都拥有一系列状态属性,如当前工作目录、指向打开文件的句柄
(在 Unix 下称为\stress{文件描述符} 即 fd,例如第 \ref{sec:shell} 节用到
的标准输入、标准输出,以及第 \ref{sec:complex} 节将涉及的\stress{标准错误
输出})等等,那么我们如何创建处于指定状态的进程呢?在 Windows 下,进程的创建
通过 \verb|CreateProcess()| 系列的函数实现,后者一般需要约 10 个参数,其中部分
参数又是包含多项信息的结构体,于是我们在创建进程时须要传递复杂的状态信息;
注意到我们本来也需要系统接口来修改进程状态属性,进行这种修改的代码相当于是在
\verb|CreateProcess()| 中重复了一次。在 Unix 下,进程的创建通过 \verb|fork()|
函数实现,其新建一个和当前进程具有相同状态属性的子进程,后者可以通过
\verb|exec()| 系列的函数把自身替换为其它程序;在 \verb|exec()| 前,子进程可以
通过普通的系统接口修改自身的状态属性,这些属性在 \verb|exec()| 时保持不变。
显然,\verb|fork()|/\verb|exec()| 只需要很少的信息,Unix 利用这一机制实现了
进程创建和进程状态控制的解耦;此外考虑在到实际应用中,创建进程时子进程往往须要
继承父进程的多数属性,\verb|fork()|/\verb|exec()| 事实上也明显简化了用户代码。

你如果了解一些面向对象编程,那么应该不难注意到 \verb|fork()|/\verb|exec()|
机制正好体现了“原型模式”的设计思路,而这一思路也启发我们思考创建系统服务进程的
方式:在 systemd 中,服务进程由其 init 程序 \verb|systemd| 创建,后者读取各服务
的配置文件,运行相应的服务程序,并根据配置文件在 \verb|exec()| 前设定服务进程的
属性;在这样的设计下,进程创建和进程状态控制的相关代码自然都要包含在其 init
中,也就是说 systemd 在概念意义上相当于是用 \verb|fork()|/\verb|exec()| 实现了
\verb|CreateProcess()| 模式的服务进程创建。借鉴之前的思路,我们可以把进程状态
控制从 init 模块完全解耦:例如在 s6 中,\verb|s6-supervise| 在 \verb|exec()|
\verb|run| 程序之前几乎不修改任何的进程属性;\verb|run| 程序几乎总是脚本(部分
实例可参考 \parencite{pollard2014}),其在设定自身的进程属性之后 \verb|exec()|
实际的服务程序。利用连续 \verb|exec()| 实现进程状态控制的技巧被形象地称为
\stress{Bernstein chainloading},因为 Daniel J.\ Bernstein 在其 qmail(首次
发布于 1996 年)和 daemontools(首次发布于 2000 年)软件中广泛应用了这种
技巧;s6 的作者 Laurent Bercot 进一步贯彻了这种技巧,他将 chainloading
的单元操作实现为一组分立的工具\cupercite{ska:execline},利用它们
可以实现一些相当有趣的需求(一个例子见脚注 \ref{fn:logtype})。

在创建服务进程时,chainloading 显然远比 systemd 的机制灵活,因为前者所用的模块
具有高内聚、低耦合的优良特点,因此易调试、易维护;相比之下,systemd 提供的机制和
同一版本下其它的 systemd 模块高耦合,所以在出现问题(如 \parencite{edge2017})
时不易替换出问题的模块。因为 chainloading 有着简洁清晰的接口,我们在须要
操作新出现的进程状态属性时可以轻松地实现相应的 chainloader,并将其集成到
系统当中而无需升级:例如 systemd 对 Linux cgroup 的支持经常被其开发者当作
systemd 的一大卖点\cupercite{poettering2013},但 cgroup 的用户接口不过是对
\verb|/sys/fs/cgroup| 目录树的操作,这在 chainloading 时很容易进行;现在已经有
一些现成的 chainloader 可用\cupercite{pollard2019},因此可以说 daemontools
式设计在对 cgroup 的支持上有天然的优势。此外,chainloader 的可组合性让我们
可以实现一些难以直接用 systemd 的机制描述的操作:例如我们可以先设定环境变量
调整后续 chainloader 的行为,然后在 \verb|exec()| 服务程序前又把环境变量
清空;一个更高级的例子见 \parencite{ska:syslogd}。

有必要指出,\verb|fork()|/\verb|exec()| 的雏形在比 Unix 更早的操作系统中已经
出现\cupercite{ritchie1980},Ken Thompson 和 Dennis Ritchie 等等人出于对实现
简洁性的追求选择了通过这一机制来实现进程的创建,因此其并不完全是 Unix 中的原创;
然而我们也看到,基于 \verb|fork()|/\verb|exec()| 及其思路可以简洁清晰地实现许多
复杂任务,这从直觉上看和第 \ref{sec:shell} 节中体现的 Unix 设计理念是符合的。
现在回到 Unix 哲学的话题:\verb|fork()|/\verb|exec()| 体现了高内聚、低耦合的
原则,方便了相关接口的协作和重用,因此我们可以勉强认为其满足上节中 Doug McIlroy
总结的前两条,尽管其不直接反映在 shell 编程中;然而这一机制并不涉及采用文本
接口与否,因此它和 McIlroy 的最后一条没有太大关系,而我认为这说明 McIlroy
对 Unix 哲学的表述并不能满意地概括 \verb|fork()|/\verb|exec()|。

\section{从 Unix 到 Plan~9}\label{sec:plan9}

1979 年发布的 Unix v7\cupercite{mcilroy1987} 已经具有了当今类 Unix 操作系统所
基于的大多数概念(如文件、管道、环境变量),以及沿用至今的许多\stress{系统调用}%
(用户空间请求内核服务的方式,如进行文件读写的 \verb|read()|/\verb|write()| 和
上节提到的 \verb|fork()|/\verb|exec()|);为了操作各种硬件设备的特殊属性,同时
避免系统调用数随着 Unix 支持的设备种类数疯狂增长,这一版本的 Unix 中引入了系统
调用 \verb|ioctl()|,后者是一个根据其参数的值操作各种设备属性的“多面手”,例如
\begin{quoting}
\begin{Verbatim}
ioctl (fd, TIOCGWINSZ, &winSize);
\end{Verbatim}
\end{quoting}
将文件描述符 \verb|fd| 所对应串口终端的窗口尺寸存入结构体 \verb|winSize| 中。
在直到这时(乃至之后少数几年)的 Unix 中,虽然有文件、管道、硬件设备等等不同的
系统资源要操作,这些操作基本都通过文件接口来实现(例如对 \verb|/dev/tty| 的
读写被内核解释为对终端的操作),换言之“一切皆是文件”;当然正如刚提到的,为了
操作各种硬件的特殊属性,出现了 \verb|ioctl()| 这样一个例外。和当今的类 Unix
系统相比,这时的 Unix 主要有两大本质区别:没有网络支持,也没有图形界面;
遗憾的是,这两项功能的加入让 Unix 越来越明显地偏离了“一切皆是文件”的设计。

Berkeley socket\cupercite{wiki:sockets}在 1983 年作为 TCP/IP 网络协议栈的用户
接口在 4.2BSD 中出现,并在 1989 年随着相关代码被其版权方置入公有领域开始成为
最主流的互联网接口;伴随 socket 而来的是一系列新系统调用,如 \verb|send()|、%
\verb|recv()|、\verb|accept()| 等等。socket 在形式上和传统的 Unix 文件类似,但
前者暴露了过多的网络协议细节,这使它的操作比后者复杂得多,其一个典型实例可见
\parencite{pike2001};此外,引入 socket 后系统调用之间出现重复,如 \verb|send()|
和 \verb|write()| 类似,\verb|getsockopt()|/\verb|setsockopt()| 和已经很丑陋的
\verb|ioctl()| 类似。在此之后,Unix 的系统调用开始不断膨胀:例如目前 Linux 有
多于 400 个系统调用\cupercite{kernel:syscalls},而相比之下 Unix v7 只有约 50
个\cupercite{wiki:unixv7};这带来的一个直接后果是系统接口整体复杂化且统一性被
削弱,导致学习成本增加。诞生于 1984 年的 X Window 系统\cupercite{wiki:xwindow}%
(即现在常说的 X 或 X11)有和 Berkeley socket 类似的问题,而且比后者更严重:%
socket 至少在形式上和文件相似,而 X 的“窗口”和其它资源则根本不以文件的形式
出现;此外虽然 X 没有像 socket 那样引入新的系统调用,但是它的可以类比于
系统调用的基本操作数比socket 相关的系统调用数大得多,而这还是在
只考虑 X 的核心模块而不包含任何扩展的情况下。

在上面的分析之后,我们自然要问,怎样在 Unix 中以符合其设计理念的方式实现对网络和
图形界面的支持?Plan~9 from Bell Labs(一般简称 Plan~9)在很大程度上正是 Unix
先驱对这一课题进行探索的产物\cupercite{raymond2003b}。之前提到,\verb|ioctl()|
和 \verb|setsockopt()| 等等系统调用是为了操作各种系统资源的特殊
属性而产生的,而这些操作似乎并不容易映射到对文件系统的操作;但从另一方面看,对
资源属性的操作也是通过用户空间和内核之间的通信完成的,只不过这种通信中传递的是
代表资源属性操作的特殊数据。在这一思路下,Plan~9 大量采用\stress{虚拟文件系统}%
来代表各类系统资源(例如网络由 \verb|/net| 代表),从而贯彻“一切皆是文件”的设计%
\cupercite{pike1995};和各种资源文件(如 \verb|/net/tcp/0/data|)关联的控制文件
(如 \verb|/net/tcp/0/ctl|)实现对资源属性的操作,不同的\stress{文件服务器}把
文件操作映射到各类资源操作,而传统的\stress{挂载}操作将目录树关联到文件服务器。
文件服务器通过网络透明的 \stress{9P 协议}进行通信,因此 Plan~9 天然地是一个
分布式操作系统;为了实现进程之间、机器之间的相对独立性,Plan~9 中每个进程
有自己独立的\stress{命名空间}(例如一对父子进程所见的 \verb|/env| 可以
互相独立,由此实现环境变量的独立性),相应地普通用户也能执行挂载操作。

利用上述机制,我们可以在 Plan~9 中只基于其约 50 个系统调用\cupercite%
{aiju:9syscalls}异常轻松地执行许多复杂任务:例如通过挂载远程机器上的
\verb|/proc| 可以实现远程调试,而通过挂载远程机器上的 \verb|/net| 可以实现
VPN 的需求;又例如通过对 \verb|/net| 设置权限可以调整用户的网络权限,而通过对
\verb|/dev/mouse|、\verb|/dev/window| 等等设置权限可以约束用户对图形界面的访问。
再回到 Unix 哲学的话题:从直觉上看 Plan~9 的设计的确体现了 Unix 哲学,但如果
说上节分析的 \verb|fork()|/\verb|exec()| 还勉强符合第 \ref{sec:mcilroy} 节中
Doug McIlroy 对 Unix 哲学的表述的话,Plan~9 所贯彻的“一切皆是文件”设计原则恐怕
很难用这一表述来概括;可见 McIlroy 的表述并不太完备,我们需要一个更好的总结。

\section{Unix 哲学:最小化系统复杂度}\label{sec:complex}

从前两节,我们已经看到 Doug McIlroy 的总结并不能满意地概括 Unix
哲学的全部;这一总结(特别是其第 1 点)的确可以认为是最主流的,
但除此之外其它的总结也有很多\cupercite{wiki:unixphilo}:
\begin{itemize}
\item Brian Kernighan 和 Rob Pike 强调将软件系统设计成易组合使用的多个小工具,
	每个工具可以相对独立地完成一类简单任务,它们组合起来使用便可完成复杂任务。
\item Mike Gancarz\footnote{有趣的是,他是 X Window 系统(见第
	\ref{sec:plan9} 节)的设计者之一。} 把 Unix 哲学总结为 9 条规则。
\item Eric S.\ Raymond 在《Unix 编程艺术》中总结了 17 条规则。
\item 除此之外也有不少其它的表述,例如上节提到的“一切皆是文件”。
\end{itemize}
我认为对 Unix 哲学的不同表述都有一定的参考价值,但它们本身也须要总结,正如上节
中提到 Plan~9 通过虚拟文件系统、9P 协议和命名空间,只用约 50 个系统调用实现了
其它系统用几百个系统调用实现的需求一样。在第 \ref{sec:shell}、\ref{sec:exec}%
--\ref{sec:plan9} 节中,我们判断一个系统符合 Unix 哲学的直观依据都是它用少而
简洁的机制和工具实现了通过其它途径实现起来更加复杂的需求,也就是说它们降低了
系统的复杂度;基于这样的观察,我认为 Unix 哲学的本质在于\stress{在几乎
满足需求的前提下最小化系统的认知复杂度},其中 3 处限制解释如下:
\begin{itemize}
\item 前提是\stress{几乎}满足需求,因为所考虑的需求往往可以分为核心部分
	(例如实现对网络和图形界面的支持)和非核心部分(例如支持 Berkeley socket
	和 X Window 系统),其中一些非核心部分可以舍弃或者用更好的方式实现。
\item 要求考虑\stress{系统}的总复杂度,因为系统中的模块之间存在交互,只考察其中
	部分模块将导致它们依赖的模块对它们的行为造成的影响被忽略:例如假设某个需求
	可以实现成 \parencite{litt2014a} 图中的两种形式,但两种实现有着相同的用户
	接口,在这种情形下我们显然不能以接口相同为由说两者符合 Unix 哲学的程度相同。
\item 明确所讨论的复杂度是\stress{认知}复杂度,因为如上述比较所示(一个更
	实际的比较可见 \parencite{github:acmetiny}/\parencite{gitlab:emca}),
	一个系统结构的优劣并不只取决于其代码尺寸,我们还须要考虑其中模块内聚和
	耦合的程度,而后者本质上是系统对人而非机器所表现的属性,对此我将在第
	\ref{sec:cognitive} 节进一步讨论。
\end{itemize}

我们来看一个比较新的例子。在以 sysvinit 为代表的一些 init 系统中,长期运行的
系统服务通过 \verb|fork()| 脱离用户 shell 的控制,实现后台运行\cupercite%
{gollatoka2011}:当用户在 shell 中运行服务程序时,该程序 \verb|fork()| 出一个
子进程,然后父进程退出,这时 shell 因为用户运行的(父)进程已经结束而等待用户的
下一命令,而子进程因父进程已结束而自动成为 init 的子进程,不再受用户 shell 的
控制。然而要控制服务的状态就要知道它的进程 ID 即 PID,上述子进程的 ID 除了存入
一个“PID 文件”之外没有太好的办法传递,而 PID 文件又是一个丑陋的机制:如果服务
进程崩溃,PID 文件将失效,而 init 系统无法得到实时的通知\footnote{事实上 init
会在其子进程退出时得到通知,但通过这一机制来监控 \texttt{fork()} 的服务程序会
制造更多的复杂度,而且并不能干净地解决问题(例如,要是服务在写 PID 文件前就
崩溃了,该怎么办?);所有其它“修复”PID 文件的尝试都被类似问题困扰,而这些
问题在使用 process supervision 时都不复存在。};此外原 PID 可能被
后续的新进程取代,使 init 系统将其它进程误认为是服务进程。

在 s6 中(其它 daemontools 类系统以及 systemd 的做法与此类似),服务进程是
\verb|s6-supervise| 的子进程,其退出时内核会立刻通知 \verb|s6-supervise|;用户
可以通过 s6 提供的工具通知 \verb|s6-supervise| 改变服务的状态,服务进程因完全
独立于用户 shell 而不再须要通过 \verb|fork()| 来进入后台。s6 的这种机制称为%
\stress{process supervision},由上述分析可见 init 系统利用这一机制可以实时跟踪
服务状态,而不用担心随 PID 文件而来的一系列问题;此外,因为在 supervision 机制下
服务进程在退出后永远由原父进程(如 \verb|s6-supervise|)重启,不像 sysvinit 机制
下在开机时由 init 的某个近亲进程创建,而重新启动时由用户 shell 创建,前一机制
下服务运行环境的可重复性要强很多。Supervision 机制的一个表面问题是服务不能像
sysvinit 下那样用 init 脚本的结束来通知 init 系统自身已经就绪,而需要另外的
机制;s6 的就绪通知机制\cupercite{ska:notify}非常简洁,且可通过
工具\cupercite{ska:sdnwrap}模拟 systemd 的机制。

Process supervision 更大的优势在于对系统日志的处理。在 sysvinit 机制下,为了脱离
用户 shell 的控制,服务进程须要将其在 \verb|exec()| 时从 shell 继承来的指向用户
终端的文件描述符重定向到其它位置(一般为 \verb|/dev/null|),于是其日志在不直接
写到磁盘时就必须通过其它方式保存。这就是 syslog 机制的来源,其让各个服务进程将
日志输出到被系统日志程序监听的 \verb|/dev/log|,这使得所有系统日志要在被混合到
一起之后再由日志程序根据指定规则分类和过滤\footnote{\label{fn:logtype}事实上,
因为 \texttt{/dev/log} 是一个 socket(准确地说,须要是一个 \texttt{SOCK\_STREAM}
socket\cupercite{bercot2015d}),日志程序原则上可以对日志来源进行有限的的判断
从而对日志流进行一定的分组,而用 Laurent Bercot 编写的工具不难实现这一需求%
\cupercite{ska:syslogd, vector2019b}。},这些操作因为涉及字符串匹配而可能在日志
量很大时成为系统的一个性能瓶颈。在 supervision 机制下,我们可以为每个服务进程
创建一个相应的日志进程\cupercite{ska:s6log}\footnote{遗憾的是 systemd 并没有这么
做,而是像 syslog 机制一样把所有日志混合到一起之后再处理。顺便提到,这里的各个
日志进程可以分别以不同的低权限用户身份运行,从而实现高度的权限分离;此外原则上
只要对这里的日志程序作些修改,就可以实现防止日志被篡改\cupercite{marson2013}的
特性,后者常被 systemd 支持者当作其专利来吹嘘。},并通过 chainloading 把前者的
标准错误输出定向到后者的标准输入,这样服务进程只须写标准错误输出即可传输日志
信息;因为各日志进程只须对相应服务(而非整个系统)的日志进行分类和过滤,这些
操作的资源消耗可以被最小化。不仅如此,利用“\stress{fd holding}”\cupercite%
{ska:fdhold}的技巧(它顺便还可以用来实现所谓“socket activation”),我们可以建立
强容错的日志信道,保证日志信息在服务进程和日志进程中任意一方崩溃重启时不丢失。

由上述分析可见 process supervision 能明显简化对系统服务及其日志的管理,
其一个非常典型的例子是对 sysvinit 机制下 MySQL 服务管理的极大简化%
\cupercite{pollard2017};因为这一机制用简洁清晰(极小化系统认知复杂度)
的方式实现了管理系统服务和日志的需求(而且还能干净利落地实现用旧机制
实现起来很麻烦的新需求),所以我认为其非常符合 Unix 哲学。

\section{Unix 哲学和软件质量}\label{sec:quality}

第 \ref{sec:intro} 节提到 Unix 诞生时的资源限制导致了其对经济和优雅的追求,
而正因如此当今不少人认为 Unix 哲学已经过时;我认为对此可以从软件质量的角度来
分析,即软件系统符合 Unix 哲学与否是不是和其质量相关。软件质量有许多定义,
其中一种\cupercite{wiki:quality}将其分为\stress{可靠性}、\stress{可维护性}、%
\stress{安全性}、\stress{性能}和\stress{尺寸}共 5 方面,显然其中后 2 方面主要
面向机器,而前 3 方面主要面向人。既然硬件资源限制是 Unix 哲学产生的最主要原因,
我们就先来看和机器更相关的 2 个方面:在硬件资源比 Unix 诞生之初丰富若干个数量级
的当今,就我们感知到的软件性能和尺寸而言,遵循 Unix 哲学是不是已经不那么重要了
呢?我倾向于给出否定的结论,对此我以目前多数用户最平常的需求之一——网页浏览为例。

随着硬件的不断升级,我们的浏览体验似乎应该越来越流畅,但我们实际感受到的却
往往并非如此:虽然下载文件的速度和观看视频的分辨率日益增长,但是我们在许多网站
上感受到的网页加载速度似乎并没有随之快速增长;这一观察或许具有一定的主观性,但
Google 的“Accelerated Mobile Pages”和 Facebook 的“Instant Articles”等等框架的
出现大概可以佐证这一现象的存在性。除此之外,浏览器占用大量内存的问题并未随着
时间的推移而消失,这在一定程度上说明除了性能问题外,尺寸问题在长远意义上
也并没有随着硬件的升级被满意地解决;这在软件领域内是一个普遍的
问题,其一个经典概括是\cupercite{wiki:wirth}
\begin{quoting}
	Software efficiency halves every 18 months, compensating Moore's law.
\end{quoting}
我认为,我们如果只满足于编写就性能和尺寸而言在同时期的硬件上刚好够用的软件,
那么或许可以不考虑 Unix 哲学;但是如果希望编写性能和尺寸不随新版本
发布逐步恶化的软件,那么 Unix 哲学仍然有其价值。

现在考虑和人更相关的 3 个方面,其中安全性将在下节专门讨论,所以我们现在着重关注
可靠性和可维护性。不可否认,当今的程序员资源和编程工具与 Unix 诞生之初有着天壤
之别,这也是当今主流类 Unix 系统能远复杂于 Multics(见脚注 \ref{fn:multics})的
原因;但另一方面,我认为这些方面的进步远不足以和 Tony Hoare 在其获得 Turing 时
的报告中总结的规律\cupercite{hoare1981}(不少计算机科学家有类似的观点)相对抗:
\begin{quoting}
	Almost anything in software can be implemented, sold, and even used, given
	enough determination.  There is nothing a mere scientist can say that will
	stand against the flood of a hundred million dollars.  But there is one
	quality that cannot be purchased in this way, and that is reliability.
	\stress{The price of reliability is the pursuit of the utmost simplicity.}
	It is a price which the very rich find most hard to pay.
\end{quoting}
Hoare 的关注点在于可靠性,但我认为可维护性在很大程度上也受此规律的制约,
复杂度和可维护性(我将开发成本看作其一方面)之间关系的一个例子可见
\parencite{rbrander2017}。下文中我将以 s6 和 systemd 为例
论证复杂度和可靠性、可维护性之间的关系。

如第 \ref{sec:coupling} 节所述,init 是 Unix 系统启动后的第一个进程,而事实上
它也是系统中整个进程树的根节点,其崩溃(退出)将导致内核崩溃\footnote{但 init
可以 \texttt{exec()},这使得 \texttt{switch\_root} 等机制成为可能;此外,s6
正是利用 init 的 \texttt{exec()} 实现了系统启动初期/关机末期相关代码和 init
系统主要子模块的解耦\cupercite{ska:pid1}。},所以它必须非常可靠;init 拥有 root
权限,因此它也必须具有高安全性。之前也提到,和 s6 的设计完全相反,systemd 的
init 模块过于复杂,而且和其它模块之间有太多太复杂的交互关系,这导致其 init
行为难以满意地控制,例如 \parencite{ayer2016, edge2017} 就是这造成实际问题的
例子。类似地,systemd 低内聚、高耦合的架构使其它模块存在和 init 模块类似的
难调试、难维护的问题:systemd 未解决的 bug 报告数量随着时间不断增长,至今没有
任何进入某种平台期(遑论开始减少)的趋势\cupercite{waw:systemd};相比之下,%
s6/s6-rc 和相关的几个软件包一旦报告任何 bug(其数量很少)几乎总是可以在一周之内
修复,而即使把其它被 systemd 在功能上模拟的项目也算进来,它们的 bug 总量也不像
systemd 那样增长\footnote{我们还可以把 systemd 和规模巨大且开发很快的 Linux 内核
对比:后者通过周期性暂停加入新特性(\stress{feature freeze})并专注修复本周期内
发现的 bug(\stress{bug converge})有效地控制了其 bug 数的增长;systemd 开发者
没有这样做,也没有采取其它项目管理手段来控制 bug 数增长,这说明他们在软件开发中
缺乏合理规划(当然这可能是因为觉得根本无法有效地修复 bug 而不引入新问题)。}。

从用户的角度看,systemd 的行为过于复杂,这使其文档只能描述其最典型的应用
场景,而大量未被开发者考虑到的情形成为真正的“corner case”(例如 \parencite%
{dbiii2016};一篇相当细致的对这类问题技术根源的分析见 \parencite{vr2015}),
这些情形下 systemd 的行为很难从文档推断;而且有时即使碰巧成功用 systemd
实现了需求,相应的配置也因 systemd 行为的影响因素太多而缺乏可重复性(例如
\parencite{fitzcarraldo2018}/\parencite{zlogic2019})。相比之下,一个熟悉
shell 编程以及和进程相关基本概念的用户花 2--3 个小时就可以从容地阅读完
s6/s6-rc 的核心文档,之后用户就可以用 s6/s6-rc 实现自己需要的系统配置,其中
如果遇到问题绝大部分都可以在很快的时间内找到原因,而且很难遇到因 s6/s6-rc
自身造成的问题\cupercite{gitlab:slewman}。此外,systemd 的行为变化太快%
\cupercite{hyperion.2019},这在其行为已经十分复杂的背景下无疑是雪上加霜;
相比之下,s6/s6-rc 和相关的几个软件包在出现(少数)破坏后向兼容性的变更时有
明确的说明,这结合相关工具良定义的行为使得更新带来的不确定性被减少到最低程度。

systemd 有着比 s6 多几乎两个数量级的开发者,而且应用了覆盖测试和 fuzzing
等等比较先进的开发方法,但是即使这样它的质量也远远不如 s6,这充分说明人力资源的
增加和编程工具的进步仍然远远不能替代对软件简洁性的追求。如果说就软件性能和尺寸
而言还可以认为 Unix 哲学重要性不如以前的话,我认为由上述分析可知,就可靠性和
可维护性而言\stress{Unix 哲学从未过时,而且比它诞生之时更加重要}:由资源限制的
消失引起的对简洁性的忽视加剧了低质量软件的横行,systemd 只是其在系统编程领域
的一种极端体现\cupercite{ska:systemd};以前资源限制强迫程序员追求简洁,现在
我们在很大程度上只能靠自律贯彻 Unix 哲学,这比以前更难\footnote{类似的
现象并不只在编程领域有,例如 1990 年代 Microsoft Publisher 等软件的
出现让普通人也能进行基本的排版工作\cupercite{kadavy2019},但由此
也助长了人们对基本排版原则的忽视\cupercite{suiseiseki2011}。}。

\section{Unix 哲学和软件安全性}\label{sec:security}

在 Edward Snowden 披露了美国的 PRISM 项目之后,信息安全成为近年受到持续关注的
话题,所以本文档专门用一节的篇幅来讨论 Unix 哲学和软件安全性的关系。如果假设软件
中的缺陷只有极少数是由怀有恶意者植入,那么安全漏洞和其它缺陷一样都基本是开发者在
无意中引入的;由此我认为可以假定软件系统的认知复杂度决定其缺陷数,因为编程工作
不过是和其它脑力劳动类似的任务,而同一个人花费等量精力制造的同类产品中错误数量
理应相近。软件系统的缺陷(包括安全漏洞和其它缺陷)随新代码而产生,随分析和调试
而消失,而分析和调试的困难程度显然取决于软件的代码量和内聚/耦合程度,也就是
系统的认知复杂度;至此我们可以看到,\stress{软件的复杂度是决定其包括安全漏洞
在内各种缺陷产生和消失的关键因素}(这大概也能解释 systemd 中未解决 bug 数为何
持续增长),因此追求简洁的 Unix 哲学对于软件安全性有着极为重要的意义。

不少软件缺陷出现的根源在于这些软件在设计上的本质缺陷,而在信息安全中很大程度上
与此对应的就是密码协议的缺陷;相应地,对于之前的纯理论分析我给出两个例子,一个
关于密码协议,另一个关于密码协议的实现。密码协议因其强数学化的特点而可以进行数学
化的分析,而信息安全领域也普遍认为没有经过充分理论分析的密码协议缺乏实用意义%
\cupercite{schneier2015};然而正是在这样的背景下,一些被广泛使用的密码协议却复杂
到了难以分析的程度,其中一个典型例子是以 IPsec 为代表的 IP 安全协议\footnote{我
认为近年新出现的 cjdns(以及后续的 Yggdrasil 等)从协议上看可能是很有潜力的
方案,因其是一个强制端到端加密(避免监控和篡改,简化上层协议)的网状网(简化
路由,且使 NAT 不再必要),使用直接从公钥生成的 IPv6 地址作为网络标识符(杜绝
IP 地址伪装),且总体设计比较简洁。有必要说明,我很讨厌 cjdns 现在的实现,后者
从构建系统到自身代码都显得过于臃肿;我甚至怀疑,几乎相同的协议如果由 Laurent
Bercot 来实现,最后的代码量可能不到现在的十分之一。}。Niels Ferguson 和
Bruce Schneier 在分析了 IPsec 之后认为\cupercite{ferguson2003}:
\begin{quoting}
	On the one hand, IPsec is far better than any IP security protocol that has
	come before: Microsoft PPTP, L2TP, etc.  On the other hand, we do not
	believe that it will ever result in a secure operational system.  It is far
	too complex, and the complexity has lead to a large number of ambiguities,
	contradictions, inefficiencies, and weaknesses.  It has been very hard work
	to perform any kind of security analysis; we do not feel that we fully
	understand the system, let alone have fully analyzed it.
\end{quoting}
并提出了以下的规则:
\begin{quoting}
	\stress{Security's worst enemy is complexity.}
\end{quoting}
类似地,David A.\ Wheeler 在讨论如何避免再次出现和臭名昭著的
Heartbleed(源于 OpenSSL 自己实现的内存分配器掩盖了其中的
缓冲区溢出问题)相似的安全漏洞时指出\cupercite{wheeler2014}:
\begin{quoting}
	I think \stress{the most important approach for developing secure software
	is to simplify the code so it is obviously correct}, including avoiding
	common weaknesses, and then limit privileges to reduce potential damage.
\end{quoting}

随着物联网的迅速发展,连接到互联网的嵌入式设备数量正在不断增长,2020 年代很可能
会成为物联网的年代,这为我们带来至少两方面的问题:第一,这些无处不在的设备上出现
的安全漏洞不仅会催生规模前所未有的僵尸网络,而且因相关设备的实际用途而可能对物理
世界的安全造成非常现实的危害,这使得安全性成为物联网首先要面对的课题之一;第二,
这些联网设备的硬件资源往往十分有限,因此软件的性能和尺寸将必然成为物联网开发中的
重要因素。正因如此,我认为\stress{Unix 哲学在 2020 年代仍将体现其重要价值}。

在结束本节之前,我想稍微离题去考察编译器后门的问题,这种后门使编译器在处理特定
程序时自动植入恶意代码:显然人们在发现异常后怀疑到编译器时会想到从干净的源代码
产生编译器本身,但如果处理编译器源代码所用的就是系统中那个脏的编译器(例如
多数的 C 编译器本身就是用 C 语言写的,因此它们可以编译自身,即\stress{自举}%
\footnote{系统引导的“booting”正是 bootstrapping 的简称,而编译器的自举是
self-bootstrapping。}),它在编译自身源代码时自动植入上述植入器的代码,这种
情况下我们该怎么办?这种极其隐蔽的后门被称为 \stress{Trusting Trust} 后门,
其从 Ken Thompson\footnote{顺便提到,他是一位国际象棋爱好者;你注意到了“预测
对方行动”的模式了吗?} 获得 Turing 奖时的报告\cupercite{thompson1984}开始广为
人知,并因这一报告的标题而得名。对抗 Trusting Trust 的一种通用思路是“Diverse
Double-Compiling”\cupercite{wheeler2009},即利用另一编译器编译可疑编译器的
干净源代码,并和后者自编译的产物比对来判断是否有 Trusting Trust 后门;另外
有一种思路是避免用编译器自举,而从底层的机器码开始逐步构建编译器\cupercite%
{nieuwenhuizen2018},我将在第 \ref{sec:benefits} 节展开讨论这一思路。

\section{Unix 哲学和自由/开源软件}\label{sec:foss}

“自由软件”\cupercite{wiki:free}和“开源软件”\cupercite{wiki:oss}这两个概念在
外延上很相近,但在内涵上又有明显的区别:前者强调\stress{(运行、)学习、分发和
改善软件的自由},后者强调\stress{使用、修改和分发软件源代码的便利}。在本文档中,
我不打算进一步分析两者的异同,而只是基于上述总结就它们对软件提出共同要求的一方面
展开讨论:显然,两者都要求用户拥有学习和改善软件源代码,从而在合理范围内调整软件
行为以满足自身需求的权利;相应地,我希望在本节表达的核心观点在于这些权利的授予
并不代表用户对软件的行为有着充分的控制,而这在极端情况下允许了\stress{形式上
自由/开源但实质上接近专有/闭源的软件项目}存在。当然,并非所有用户都有能力学习和
改善软件源代码,所以本节对所涉及软件项目的比较都从同一位具有适度
计算机科学和软件工程背景的用户的角度出发。

我们知道,只发布被混淆源代码的软件是没有资格被称为自由/开源软件的,因为混淆让
源代码难以被人理解,或者说增加了源代码的认知复杂度;另一方面,从之前的分析,我们
也知道低内聚、高耦合的软件系统也具有很高的认知复杂度,而有些自由/开源软件项目也
受此问题的影响,例如当今主流的开源浏览器 Chromium、Firefox,以及之前多次提到的
systemd。可以注意到,用户对后面这些软件行为的控制被明显削弱了:对于 Chromium
和 Firefox,其典型标志是在出现蔑视用户需求的更新(如 \parencite{beauhd2019,
namelessvoice2018})时,用户除了向其开发团队请愿之外少有其它选择\footnote{%
Firefox 有 Waterfox、Pale Moon 等等替代品,但这些替代品因人力资源的限制而在
安全更新等方面落后于 Firefox\cupercite{hoffman2018}。};对于 systemd 而言,
其典型标志则是用户在遇到各种从一开始就不该存在(见第 \ref{sec:quality} 节)
的“corner case”(例如 \parencite{ratagupt2017})时,在其开发者设法修复之前
只能想方设法绕过问题,而且开发者还可能直接拒绝考虑相关需求(例如 \parencite%
{akcaagac2013}/\parencite{junta2017}\footnote{借助早在本世纪初就已存在的
chainloading 技巧(见第 \ref{sec:complex} 节),我们可以完全避免 systemd 中
\texttt{journald} 所用的二进制日志格式,同时却比 \texttt{journald} 更加简洁、
可靠地实现后者实现的绝大部分用户需求。此外即使抛开 \texttt{journald} 本身多余
与否这一点,我也至今没有看到日志信息强制通过其转发到 syslog 日志服务的任何
技术优势,而加入让 syslog 服务直接监听日志信息的功能对 systemd 开发者似乎
一点都不难:只要允许设定 \texttt{journald} 不监听 \texttt{/dev/log} 即可。}
和 \parencite{freedesktop:sepusr}\footnote{然而其不支持在无 initramfs 时
分开挂载 \texttt{/usr} 的论据可谓非常薄弱\cupercite{saellaven2019a}。})。
事实上,用户对上述软件的控制还不如对一些提供源代码但限制分发的软件,例如第
\ref{sec:wib}--\ref{sec:howto} 节中将提到的 Chez Scheme 的旧版本。由此可见,
从 s6 等允许高度控制的软件和旧 Chez Scheme 等允许充分控制的软件,到 systemd
等只允许很有限控制的软件和传统的专有/闭源软件,\stress{从用户对软件
行为的控制上看,自由/开源和专有/闭源的界限已经开始模糊}。

上述分析是从纯技术方面入手的,用户对自由/开源软件行为的控制被削弱的确主要是因为
这些软件低内聚、高耦合的架构,然而我认为其中有一个重要的例外:接下来,通过和专有
软件的对比,我将论证 \stress{systemd 在开源界内开启了采用专有式手段进行推广的
先河}。systemd 的支持者和反对者多数都同意 systemd 成为主流 Linux 发行版中默认
init 系统的最重要转折点是它在 Debian 发布的“jessie”版本中成为默认\cupercite%
{sfcrazy2014, paski2014}\footnote{这两次投票中所用判定规则在原场合下的合理性都
受到了争议\cupercite{dasein2015, coward2017},不过无论如何 Debian 一方的决定
是否合理并不影响 systemd 开发者自身行为的不义性。类似地,elogind 的存在不能否定
systemd 开发者期望 \texttt{logind} 和 systemd 捆绑的既定事实,因此也不能否定
上述的不义性,而相同的结论对下文所述的 udev 以及 kdbus 相关事件也成立。},而且
同意造成后者的最主要原因是 GNOME~3 开始依赖 systemd 中 \verb|logind| 所提供的
接口\cupercite{bugaev2016}。然而,虽然名义上被依赖的只是 \verb|logind| 接口%
\cupercite{vitters2013},systemd 开发者很快明确表示 \verb|logind| 一开始就是
设计成和 systemd 捆绑在一起的\cupercite{poettering2013},这造成了 GNOME~3 对
systemd 的事实依赖;另一方面,我至今没有看到任何可信的关于 systemd \verb|logind|
相对于 2015 年出现的 elogind 优势的分析,因此 systemd 开发者是在明知没有
技术优势的前提下实施了 \verb|logind| 和 systemd 的捆绑。

在此之后,systemd 开发者又企图把 udev 捆绑到 systemd\cupercite{poettering2012},
并试图通过推动 kdbus 进入 Linux 内核\footnote{在其他内核开发者的提问(例如
\parencite{lutomirski2015})之下,其技术理由\cupercite{hartman2014}逐渐被证明
站不住脚,而 kdbus 最终也没有进入内核。}来增加 eudev 项目独立实现和 udev 兼容
接口时的开发成本\cupercite{poettering2014}。考虑到 systemd 开发者对其之前承诺%
\cupercite{poettering2011a, sievers2012}的明显背弃,以及他们在明知缺乏技术优势%
\cupercite{cox2012}的前提下不顾 eudev、mdev 等等类似项目执意推动 kdbus 的行为,
我觉得完全可以认定 systemd 开发者有意实施了典型的“\stress{embrace、extend、%
extinguish}”\cupercite{wiki:eee}(EEE)手段,导致了非必要的\stress{提供商依赖}:
\begin{itemize}
\item 在自己的项目中开发可被下游项目使用的技术,
	这类技术可能扩展了现有的类似技术。
\item 游说下游项目使用上述的技术(在此过程中可能作出低耦合的虚假承诺);在有
	扩展功能时推动这些功能的使用,从而为和自己竞争的其它项目制造兼容性问题。
\item 自己的项目形成事实标准之后,在明知没有技术优势的背景下将上述技术
	捆绑到自己的项目,从而排挤其它“不兼容”的项目。
\end{itemize}
不可否认的是开源界不是世外桃源,其中充满了纷争乃至所谓“圣战”,但据我所知以前
没有任何一次纷争涉及的开发者像 systemd 开发者这么明目张胆地使用 EEE 那样的手段%
\footnote{例如 GNU 软件常被抨击为过度臃肿,并因特性的堆砌而排挤了更简洁的类似
软件,然而这些软件多数有较强的的可替代性;GCC 或许是被下游项目硬性依赖最多的 GNU
软件之一,但一方面其特性集似乎不容易低耦合地实现(例如和 GCC 竞争的 LLVM 在架构
上似乎并不比其好太多),另一方面我们没有确切证据其表明开发者在无技术优势的前提下
有意以紧耦合的方式加入新特性。}。我认为自由/开源软件的开发者应该具有比专有/闭源
软件的开发者更高的道德标准,因此这样的手段虽然并不违反各大开源许可证\footnote%
{顺便提到,TiVo 化\cupercite{wiki:tivo} 也不违反多数开源许可证。},但却显得比
专有/闭源软件社区中的同类行为更加卑鄙,或者如 Laurent Bercot 所言\cupercite%
{bercot2015a, bercot2015b}(我称之为“\stress{自由/开源的恶意臃肿软件}”):
\begin{quoting}
	systemd is as proprietary as open source can be.
\end{quoting}

尽管 systemd 的闹剧尚未收场(第 \ref{sec:devel}--\ref{sec:user} 节将讨论如何加速
其进程),我们仍然可以对其反映出的问题进行反思:这场闹剧的根源在哪里,如何防止
这样的闹剧再次出现?如第 \ref{sec:quality} 节所述,我认为 systemd 闹剧在技术
上的根本原因是由硬件资源限制的消失引起的对软件简洁性的忽视,而这导致的低内聚、
高耦合在具有关键意义的系统软件中被其开发者“创新性地”和 EEE 的手段结合之后就造成
了现在这样的提供商依赖;为了避免这样的闹剧再次上演,我们须要意识到\stress{自由/%
开源软件应当牵手 Unix 哲学},因为只有这样才能斩断 EEE 在开源界借低内聚、高耦合
之尸还魂的途径。有一种观点(例如 \parencite{bugaev2016})认为 systemd 及其中
模块正好符合 Unix 哲学,但现在我们应该已经很明白这种观点并不正确,由此应当
吸取的教训是在讨论 Unix 哲学时必须清醒地认识到我们讨论的是\stress{系统的总
复杂度}:和原来通过 shell 充分重用已有工具的系统相比,在看到 systemd 高内聚、
低耦合的架构,它复杂的外部依赖关系\cupercite{github:sdreadme}(增加了 systemd
和外部的耦合),及其对系统中已有工具中功能的重新实现\cupercite{wosd:arguments}%
\footnote{这些重新实现少有比原来做得更好的,其中一部分(如 \parencite%
{wouters2016, david2018})甚至可谓灾难;对此 systemd 支持者惯用的说辞是
“这些功能可以关掉”(全然不顾其有无技术优势的问题),以及“你行你上”(完全
无视“谁污染谁治理”\cupercite{torvalds2014}的原则)。}(忽略了协作和重用)
时,我们会认为它“小巧、极简、轻量级”还是“庞大、混沌、冗余、
资源密集”\cupercite{poettering2011b}?

在本节末尾我希望强调,systemd 的闹剧对开源界是一场巨大的挑战,但同时也是重要的
机遇:如上文所述,它让我们清楚地看到盲目忽视 Unix 哲学造成的严重后果,不吸取
这一惨痛教训的结果将必然是“亦使后人而复哀后人也”;systemd 注定要被钉在开源界的
耻辱柱上,但另一方面它也促使我们再次审视那些优秀的软件系统(包括 Plan~9 和
daemontools 等等“非主流”软件),并从中学习如何在实际工作中贯彻 Unix 哲学。
关于这方面的更多细节,我将在第 \ref{sec:devel}--\ref{sec:user} 节中进一步讨论,
在这里我只就自由/开源软件的话题做最后两点补充:第一,追求简洁能\stress{为志愿者
节省时间精力}(这些没有固定资金支持的人在开源界中大量存在,而且做出了巨大的
贡献),让他们更容易专注于最有意义的项目,这在新需求随着技术进步不断涌现的当今
尤为重要;第二,简洁清晰的代码会自然地鼓励用户参与开发过程,\stress{增加对
自由/开源软件的有效复查},从而促进软件质量的提升,这或许是从源头上防止
Heartbleed 灾难重演、使 Linus 规则\cupercite{wiki:eyeball}
\begin{quoting}
	Given enough eyeballs, all bugs are shallow.
\end{quoting}
成为现实的一种途径。

\section{极简主义实践:开发者视角}\label{sec:devel}

我在上节末尾提到追求简洁能节省时间精力,这事实上是一种简化的表述:为了让软件
简洁,开发者须要花费相当多的时间精力进行架构设计,因此在短期内其可见的产出或许
不如用脏乱差方案快速解决问题的开发者;然而一旦实现了相同的需求,和臃肿晦涩的代码
相比,简洁清晰的代码会具有更高的可靠性、可维护性和安全性,因此从长远上看将节约
开发者在整个软件生命周期内花费的时间精力。在商业开发中,有时为了抢占市场,迅速
推出新特性,须要把简洁性放到次要地位来实现像 Facebook 那样的“move fast and break
things”,但我认为追求简洁从长远看仍然应该是常态:一方面,抢占市场在开源界是比较
少见的需求,因为原则上本身优秀的项目可以凭质量取胜\footnote{\label{fn:plan9}%
这里说“原则上”是因为有一些微妙的“例外”,我以没能取代 Unix 的 Plan~9 为例说明。%
Plan~9 至今有 4 个正式发布版\cupercite{wiki:plan9},其中 1992 年的第 1 版只对
大学发布,1995 年的第 2 版只对非商业用途开放,只有 2000、2002 年的第 3、4 版是
真正开源的,这让它完全错过了通过自由传播来获得影响力的最佳时机。另一方面,在
2000 年 Plan~9 开源后,其圈子内的人们因为较为激进的极简主义(从 \parencite%
{catv:hsoft} 可见一斑)而根本不去实现一些需求,例如带有 JavaScript 功能的网页
浏览器;我认为这些需求的确丑陋,但不去实现它们的后果自然是用户很难适应,毕竟非
平滑过渡在软件领域是一个具有普遍性的难题。此外,近年来上层开发越来越不注重简洁性
(你如果自行编译过构建 TensorFlow 所用的 Bazel 程序,也许就会对此有直观的体会;%
Google 明明聘请了 Ken Thompson 和 Rob Pike 等等人,其软件系统却仍然如此臃肿,
对此我感到费解),这也使 Plan~9 和“现代”的需求渐行渐远,而我编写本文档的一个
重要目的也是提升大家对极简主义的认知。},所以经常要抢占市场的项目令我不得不联想
到低的软件质量和以 embrace、extend、extinguish 为代表的卑劣手段;另一方面,用来
抢占市场的特性多数不会被迅速丢弃,因此为了保持软件系统的可维护性,重构是迟早
要进行的。由此可知在软件开发,特别是开源开发中,追求简洁的 Unix 哲学
的确是一条值得贯彻的原则,而本节就将讨论如何实际贯彻这一原则。

在进入具体细节之前,我们有必要确定总体的实践原则:既然要追求简洁,我们就应当
树立以简洁为荣的观念,以用最小可行程序\cupercite{armstrong2014}实现既定需求
作为编程能力的标准之一,而非只看代码产量;为此我们应当牢记 Ken Thompson 的话,%
\stress{以生产负代码\textmd{\cupercite{wiki:negcode}}为荣、经常考虑重构}:
\begin{quoting}
	One of my most productive days was throwing away 1000 lines of code.
\end{quoting}
只有追求简洁的态度显然不够,我们还需要具体的方法来提升编写简洁代码的能力,
其中\stress{观摩现有的优秀软件项目以及相关讨论}是实现自我提升的一条重要
途径,而我个人建议从 Laurent Bercot 的项目\cupercite{ska:software}、Plan~9%
\cupercite{wiki:plan9}、suckless 系列项目\cupercite{suckless:home}和 cat-v
网站上关于软件的讨论\cupercite{catv:hsoft}\footnote{我认为 cat-v 网站明显地
较为激进,因此建议理性看待其内容。}开始学习。在总体较为合理的基础(例如 Unix)
之上设计简洁系统的关键在于\stress{巧妙地重用现有的机制},这需要对这些机制
本质属性的深刻理解,其中一些典型的例子如下,它们尤其值得我们学习:
\begin{itemize}
\item 第 \ref{sec:plan9} 节所述的将对系统资源属性的操作看作用户空间和
	内核之间传递特殊数据的通信,这种通信可以映射到对控制文件的操作,
	从而避免使用 \verb|ioctl()| 系列的系统调用。
\item qmail\cupercite{bernstein2007} 借助 Unix 中的用户权限机制实现其访问
	控制,借助文件系统实现其邮箱别名机制,并利用 \verb|inetd| 的思路(其实是
	UCSPI\cupercite{bernstein1996})实现了传输层和用户层代码的分离。
\item LMDB\cupercite{wiki:lmdb} 借助 \verb|mmap()|、写入时复制等等
	机制用几千行代码实现了具有优良属性和优异性能的键{--}值存储,
	其中通过基于 B+ 树的分页跟踪机制避免了垃圾回收。
\item 在讨论发布{--}订阅式(总线式)消息传递的实现策略时,%
	Laurent Bercot 指出\cupercite{bercot2016}其中传输的数据需要
	基于引用计数的垃圾回收,而 Unix 的文件描述符正好满足这一需求。
\end{itemize}

刚刚提到,在总体较为合理的基础上设计简洁系统的关键在于巧妙重用,但如果遇到
不合理的地方怎么办呢?事实上,当今主流的类 Unix 系统中从底层到上层都有很多
不合理之处,而有眼光的人并未对这些问题视而不见,例如:
\begin{itemize}
\item 如第 \ref{sec:plan9} 节所述,BSD socket 机制和 X Window 系统是 Unix
	走向臃肿的里程碑,而 Plan~9 正是 Unix 先驱对以此为代表的问题深入思考的
	产物;类似地如第 \ref{sec:complex} 节所述,process supervision 正是
	Daniel J.\ Bernstein 等人对系统服务及其日志的管理方式进行反思的产物。
\item 当今通用的 C 标准库接口并不理想\cupercite{ska:djblegacy},而作为可移植标准
	的 POSIX 只是对类 Unix 系统中已经存在的行为加以统一,而不论其美丑。Bernstein
	在编写 qmail 等软件时仔细审视了这些接口,并通过对系统调用(如 \verb|read()|%
	/\verb|write()|)和一些不太糟糕的标准库函数(如 \verb|malloc()|)的小心封装
	定义了一组质量高得多的接口;这组接口被其他开发者从 qmail 等软件中分离出来,
	并有了多个后继,其中我认为最好的是 skalibs\footnote{其作者指出\cupercite%
	{ska:libskarnet}在很多情况下,用 skalibs 编写的程序静态链接产生的可执行
	文件要比用标准库或其它工具库编写的类似物小一个数量级。顺便提到,动态链接
	原先是为了解决臃肿的 X Window 系统占用空间过多的问题而产生的,类似问题
	随着硬件条件的改善而已经被极大减轻,因此动态链接造成的各种麻烦使得关注
	简洁性的开发者比以前更加倾向于使用静态链接\cupercite{catv:dynlink}。}。
\item 类 Unix 系统中普遍使用的 Bourne shell(\verb|/bin/sh|,及其后继
	\verb|bash|、\verb|zsh| 等等)有着相当古怪的接口\cupercite{ska:diesh},
	例如以 \verb|$str| 所存储的字符串为名的变量的值一般要借助危险的 \verb|eval|
	来访问,因为我们不能使用 \verb|$$str| 或 \verb|${$str}|;但这些怪癖并不是
	所有 shell 的通病,例如 Plan~9 的 \verb|rc| shell 就避免了
	Bourne shell 的多数问题\cupercite{vector2016b}。
\end{itemize}
类似地,我们应该养成\stress{用批判的眼光看待现有软件}的习惯:例如传播软件自由
的 GNU 项目生产了许多质量平庸的软件\cupercite{bercot2015c},格外关注安全性的
OpenBSD 对 POSIX 的支持甚至比 macOS 的还糟糕\cupercite{bercot2017},曾是开源界
“杀手应用”的 Apache 实现臃肿低效;这些问题绝不影响上述项目的重大意义,但我们
不应就此安于现状。进一步地,我们应当冷静地看待当前的成果和潮流,\stress{关注
本质而非表象}:例如在读到这里的时候,你应该已经初步了解 systemd 中
许多特性的本质是什么,它们和 systemd 的架构有几分必然联系;在第
\ref{sec:homoiconic} 节中,我将用另外一些例子演示我对“语言特性”
这一概念的理解,后者或许能帮助你换一个角度看待当今流行的一些程序语言。

有必要指出,本文档中提到的做出重大贡献的人自己也会犯错:例如“silly qmail
syndrome”\cupercite{simpson2008}的本质原因是 Bernstein 没有正确实现邮件队列、
邮件分拣(写入端)和邮件传输(读取端)构成的 SPOOL 系统,后者是操作系统中
比较有用的一个概念;Tony Hoare 在其 Turing 奖报告\cupercite{hoare1981}中建议
编译器采用单步处理的方案,然而我们将在第 \ref{sec:wib} 节中看到多步处理可以
做得比单步处理更简洁、清晰、可靠;Plan~9 的设计者出于对硬件价格的考虑而在
Bell 实验室采用多台终端连接到少数中心服务器的网络架构,并且提到 Plan~9 完全
可以在个人电脑上使用但他们不认同这种做法\cupercite{pike1995},这样的设计
在硬件十分廉价、对中心化服务器信任度下跌的当今显然已经不太适用。那么
既然任何人都有可能犯错,我们应该相信谁?我认为关键在于两点:
\begin{itemize}
\item \stress{具体问题,具体分析}:考察论据和论证,注意论据是否适用于当前的
	应用场景\footnote{类似地,本文档有意偏离了学术界中避免引用 Wikipedia 的
	惯例,这也是本文档只有“参考资料”而非“参考文献”的原因;其理由在于我希望为
	读者提供(多少)更加生动活泼,而且不断更新的材料。}。例如众所周知 C 编程中
	应当避免使用 \verb|goto| 语句,但包括 Linux 内核在内的许多项目都在异常处理
	中大量使用“\verb|goto| chain”\cupercite{shirley2009},因为避免 \verb|goto|
	是为了防止复杂的来回跳转把代码变为烂面条,而相反 goto chain 不仅不损害代码
	的可读性,而且具有比所有等价写法更好的可读性。类似地,第 \ref{sec:complex}
	节中将 Unix 哲学的本质总结为复杂度问题不是为了好看,而是为了
	在用 Unix 哲学判断系统优劣时正确领会其精神。
\item \stress{广泛调查,兼听则明}:倾听各方观点,从而尽量使自己了解问题的
	全貌。例如 systemd 的暂时“成功”在很大程度上要归因于 Linux 圈大多数人
	只对 sysvinit、systemd、Upstart 和 OpenRC 有比较充分的了解,而几乎忽略了
	daemontools 式的设计及其潜力;在充分了解各种 init 系统的设计,以及各方支持
	者、反对者对它们的比较之后,你自然会明白应当选择怎样的 init 系统。类似地,
	部分 systemd 支持者必然会不遗余力攻击本文档的观点,而我只希望你充分阅读
	本文档和相关的参考资料,然后结合自己了解的各方观点形成自己的结论。
\end{itemize}

在结束本节之前,我希望额外讨论一些关于 systemd 的问题。首先,systemd 的闹剧并未
收场,而我们只有加紧完善其替代品、使它们在实际实现的需求上趋于完备才能加速这一
进程,因此我呼吁所有对 systemd 的糟糕属性感到失望的开发者参与或关注这些替代品的
开发。其次,尽管我们有 eudev、elogind 等项目,但是它们的母项目本身代码质量平庸
(否则很难被 systemd 捆绑),因此在这上面和有充足人力资源的 systemd 项目赛跑必然
导致一定的劣势;反过来看,我们应当着重支持 mdevd、skabus 等等从头开始、追求简洁
清晰的替代项目,在标准上用“良币驱逐劣币”。最后,“\stress{己所不欲,勿施于人}”,
我们在参与陌生的开源项目时应当对其习惯保持必要的尊重,避免像 systemd 开发者
那样强加自身观点于他人的倾向,这在和 init 系统无关的项目中同样重要:例如因为
多方面因素的影响,不同人追求简洁的程度不一样,但只要这样的个人选择是在充分
调查、仔细权衡之后作出的,而且不侵犯他人的选择权,我们就有必要尊重这种选择。

\section{极简主义实践:用户视角}\label{sec:user}

上节提到,对开发者而言,追求简洁在短期内可能不利,但从长期看能节省时间精力,
而类似的结论对普通用户也成立:例如 Windows 固然有“用户友好”的图形界面,但一旦
涉及一点稍微复杂的任务而没有现成的工具时,在 Windows 下我们仍然要借助其和类 Unix
系统中 shell 类似的一些工具(批处理、VBScript、PowerShell,或者 Python 等跨平台
语言);由此可见,如果你希望能“自己动手,丰衣足食”,那么你迟早须要学习使用那些
不“用户友好”的工具,而我们在第 \ref{sec:shell} 节中已经比较直观地看到通过 shell
把 Unix 工具组合起来产生的巨大威力,所以这样的学习的确会带来丰厚的回报。我认为
shell 并不难学习,关键在于理解其基本用法,而不是如上节提到的 Bourne shell 的
各种怪癖;我建议在初步学习 Bourne shell 之后认真学习一遍 \verb|rc|\cupercite%
{github:rc},因为后者简明地体现了 shell 编程的核心,这样再回过来看
Bourne shell 时就知道哪些地方相对次要了。

进一步地,正如 shell 和图形界面的关系一样,简洁的软件系统和复杂的软件系统之间
有着类似的关系,这里我以 RHEL/CentOS 和 Alpine Linux\footnote{顺便提到,如果
多数发行版应该被称为“某某 GNU/Linux”\cupercite{instgentoo:interj},那么 Alpine
是不是应该被称为“Alpine BusyBox/musl/Linux”?}为例。Red Hat 的系统像 Windows
一般大而“全”,在严格按照其开发者设想的方式使用时一般能如预期地工作,但其系统中
为了使这些“设想的方式”更加“用户友好”而进行了许多额外且缺少文档的封装;由此产生
的问题是出现故障时系统因其高复杂度而难以调试,而且用户在有特殊需求时不得不
想方设法绕过上述的封装,还要担心绕过时对其它系统组件行为的潜在影响\cupercite%
{saellaven2019b}。与此相反,Alpine Linux 采用 musl、BusyBox 等等简洁的组件构建其
基础系统,并且尽量避免不必要的封装和依赖,这使其虽然明显不如 Red Hat 系统“用户
友好”,但是对于有基本 Unix 背景的用户而言非常稳定可靠、容易调试和定制。显然,
这让我们联想到第 \ref{sec:quality} 节中 systemd 和 s6/s6-rc 的对比,由此可以
直观地感受到软件复杂度不仅影响开发者,而且对用户体验也有明显的影响,正如
Erlang 语言的设计者 Joe Armstrong 所述\footnote{我清楚地知道原文是对
面向对象编程的评论,事实上隐含状态如全局变量一样是引入耦合的因素,
因此对状态缺乏隔离的面向对象系统有和复杂软件系统异曲同工的
高耦合问题,这一问题在系统中有多层封装、继承时尤为严重。}:
\begin{quoting}
	The problem with object-oriented languages is they've got all this implicit
	environment that they carry around with them.  You wanted a banana but what
	you got was a gorilla holding the banana and the entire jungle.
\end{quoting}
如第 \ref{sec:foss} 节所述,我认为软件复杂度对用户体验的影响在自由/开源软件系统
中尤为重要,因为其内部构造对用户开放,从而给了用户自行进行调试和定制的可能。

基于上述原因,我认为\stress{即使是普通用户,在选择软件系统时也应注意考察其
复杂度},优先采用 musl、BusyBox、s6/s6-rc、Alpine Linux 以及 \verb|rc|、vis、%
abduco/dvtm 等等简洁软件;在简洁但问世不太久的软件(如 BearSSL)和复杂但经过
相对仔细复查的软件(例如 LibreSSL/OpenSSL)之间选择时,只要前者的作者有良好
的职业记录而且认为其软件足够实用,优先采用前者;不得不在多套臃肿的软件之间选择
时,优先采用相对简洁且经过实践检验的软件,例如在使用基于 systemd 的系统时尽量
使用通用工具、避开 systemd 中的重新实现。另一方面,即使轻量级替代品不足以实现
自己的主要需求,我们也可以支持或关注它们,以便在其功能足够完善时尽早迁移:例如
我是在 s6-rc 首次发布前一年多开始注意到 s6 项目的,在 s6-rc 推出(从而使系统支持
服务间依赖和短期运行的 init 脚本)之后,我便开始准备将自己的系统向 s6/s6-rc
迁移,并在其推出一年后实现了首个系统的迁移。s6 相关的软件在 systemd 成为 Debian
的默认 init 系统时诚然不能满足一般用户的需求,这也是 systemd 成功获得其统治地位
的一个技术原因;但目前 s6/s6-rc 已经非常接近满足一般用户需求的标准\cupercite%
{vector2018a, vector2019a}\footnote{此外我猜想所有在实际应用中足够有用的
systemd 功能都可以在基于 s6/s6-rc 的基础架构中实现,其中许多功能的代码量
将不到 systemd 中对应功能的 $1/5$。},同时还拥有 systemd 开发者声称
却未能兑现的优良属性\cupercite{poettering2011b},因此我请求所有
感兴趣的用户积极关注和参与 s6/s6-rc 和相关软件的开发。

有时用户对软件系统的选择权受限于工作需求等等因素,而我认为在这种情况下用户可以
在当前约束下以尽量简洁清晰的方式使用其系统,并注意为更理想的解决方案留出余地,
从而\stress{最小化未来可能发生的迁移所需的成本}:例如多数 Linux 发行版的系统维护
脚本用 Bourne shell 而非 \verb|rc| 写成,而用户可能须要对其中一些进行定制(这
也是上文中没有建议只学习 \verb|rc| 而跳过 Bourne shell 的原因),此时用户可以把
定制的部分写成简洁清晰且尽量符合 \verb|rc| 用法的形式,以备在合适时把被定制的
脚本用 \verb|rc| 重写;又例如我因为工作原因须要和使用 systemd 的 CentOS 7
打交道,我选择的策略是在自己可控的机器上完成大部分工作,并尽量
在虚拟机中以最简的方式操作 CentOS 7,从而减少对其依赖。

如果你因为自己惯用的软件系统(Linux 发行版或桌面环境)而不情愿地使用 systemd,
我建议你积极尝试简洁的软件,逐渐摆脱那些和 systemd 强耦合的软件,并从此牢记
systemd 闹剧教给我们的教训:在这并不平静的开源界,\stress{只有掌握了简洁性这件
秘密武器,才能在遇到和 systemd 类似的危险陷阱时将命运掌握在自己手里},从容躲避
而不是束手就擒\footnote{即使对从源代码编译一切、可定制性很强的 Gentoo,希望使用
GNOME~3 又不用 systemd 的用户也曾长期必须翻越重重障碍\cupercite{dantrell2019},
而希望不借助 initramfs 分开挂载 \texttt{/usr} 的用户在面对盲目采取 systemd
做法的开发者\cupercite{saellaven2013}时至今须要自行维护补丁集\cupercite%
{stevel2011}。}。为此我认为用户有必要特别警惕系统中诸如 GNOME 的那些臃肿的组件,
在发现其有和可疑项目紧耦合的趋势时及时考虑如 Fluxbox、Openbox、awesome、i3 等等
轻量级替代品:在那些臃肿项目和恶意臃肿软件开始耦合之后,其开发者即使毫无恶意也
未必能承担清除自己项目中被感染部分所需的成本\cupercite{vitters2013},这是臃肿
项目的高复杂度造成的自然结果。进一步地,我认为应当对主要开发者受诸如 Red Hat
这样同一家商业公司资助的那些过度臃肿的软件项目\cupercite{saellaven2019a}保持
必要的警惕\footnote{我们也有必要进一步对由商业公司主导的复杂标准保持警惕,例如
现在由各大浏览器厂商及其背后公司主导的 HTML5;此外我也要对虽在 Red Hat 却勇于和
systemd 开发者斗争的 BusyBox 开发者\cupercite{vlasenko2011}表示由衷的敬佩。}:
因为从 Chromium 和 Firefox 的例子(见第 \ref{sec:foss} 节)可见开源界的公司一样
可能牺牲用户利益来换取商业利益,所以如果这些臃肿项目的开发者受到指使而合谋实施
embrace、extend、extinguish ,我们将面临腹背受敌的局面;退一步说,即使 EEE
行为不是受公司指使,纵容员工采用下作手段制造垄断、在自身和开源界之间制造利益
冲突的公司也是十分可鄙的,对此我们应当用自己的包管理器投票以表达对它们的唾弃。

你可能会注意到,本节标题中写明了“用户视角”,但本节所涉及部分软件的简洁性往往在
用户具有一定背景时才能充分理解;事实上,第 \ref{sec:intro} 节已经提到 Unix 被
设计为提升程序员工作效率的系统,而我们也看到 Unix 的确是在其内部机制被用户充分
了解时工作得最好。然而多数用户用于理解其所用系统运行机制的精力毕竟有限,这是不是
意味着只有程序员才能用好类 Unix 系统呢?我觉得并非如此,只是之前没有相关背景的
Unix 用户须要克服对编程的排斥,\stress{从好的工具和教程入门},这并不会花费过多的
精力:例如假设在粗略学习 Bourne shell 之后仔细阅读 \verb|rc| 的文档,然后再回顾
Bourne shell,即使新手在进行充分练习的前提之下也能用一两天的时间学会基本的 shell
编程;如果再用另外一两天时间学习和 Unix 进程相关的基本概念,这位用户就可以开始
学习使用 s6/s6-rc 了。入门之后,用户在学习中如果能注意以下几点,将能事半功倍:
\begin{itemize}
\item \stress{注意判断所学知识的重要性}:时常思考自己学到的知识中哪些具有
	更强的\stress{普适性}和\stress{有用性},正如 \parencite{dodson1991}
	所述(我们将在第 \ref{sec:boltzmann} 节进一步讨论这一论断):
\begin{quoting}
	Self-adjoint operators are very common
	and very useful (hence very important).
\end{quoting}
\item \stress{勇于求助,善于求助}:在遇到自己难以解决的问题时充分
	利用开源社区的力量,但在求助时要注意尽量简明、可重复地表达
	遇到的问题,并务必给出自己想到的思路。
\item \stress{避免过度编程}:如上节所述,以简洁为荣,追求用简洁清晰的方法
	解决问题;在日常工作中避免小题大做,小批量的操作可以考虑手工完成。
\end{itemize}

在结束本部分之前,我希望引用 Dennis Ritchie 的名言:
\begin{quoting}
	Unix is very simple, it just needs a genius to understand its simplicity.
\end{quoting}
这句话可以说是既对也错,说对是因为用好 Unix 需要对其运行机制的本质理解,说错
是因为勇于学习的人在合理的引导之下并不难达成这一目标:我认为高效学习 Unix
的本质在于\stress{最小化完成实际任务所需学习过程的总复杂度},其关键在于
培养降低总复杂度的习惯;这种习惯将个人的工作和生活与 Unix 哲学联系起来,
而我将在第 \ref{sec:worklife} 节中阐述这种习惯的重要意义。


\newpart
\section{Hilbert 第 24 问题:数学中的极简主义}\label{sec:hilbert}

1900 年,David Hilbert 发表了一个包含 23 道未解数学难题的清单,这些将被称为
“Hilbert 问题”的难题影响了整个 20 世纪的数学家,而且仍在产生重要影响;2000 年,
有人在 Hilbert 的笔记中发现了一个因故被排除在原先清单之外的附加问题,而我们就
从这个第 24 问题\cupercite{wiki:hilbert}开始本部分的内容。Hilbert 第 24 问题
本质上寻求的是对数学证明简洁性的形式化标准,以及论证给定证明是相应定理最简证明
的方法;由此我们已经可以注意到在数学中同样存在对简洁性的追求,而这种追求其实也
存在于其他伟大数学家身上,例如 Paul Erdős 经常提起“THE BOOK”\footnote{其在现实
世界中的一个“近似”,《Proofs from THE BOOK》\cupercite{wiki:thebook},在 1998
年被献给 Paul Erdős,后者在此书编写过程中提出许多建议,但在其出版之前就逝世
了。},一本假想的上帝用来保存最简证明的书,并且在 1985 年的一场讲座中说到:
\begin{quoting}
	You don't have to believe in God, but you should believe in THE BOOK.
\end{quoting}
对简洁性的追求似乎也并不是大师们的专利,因为 Edsger W.\ Dijkstra 曾经说过:
\begin{quoting}
	How do we convince people that in programming simplicity and clarity --
	in short: what mathematicians call ``elegance'' -- are not a dispensable
	luxury, but a crucial matter that decides between success and failure?
\end{quoting}
由此可见数学家在某种意义上比程序员更加极简主义,而我们将在第
\ref{sec:science} 节中见到极简主义在数学中的一种极端体现。

在 Hilbert 关于第 24 问题的笔记中,他试图将一族特定的证明归约为一族主要由
特定类型元素组成的序列,并通过序列的长度来判断简洁性;从程序语言的角度,这不难
推广\footnote{这种对应关系可以通过\stress{Gödel 数}\cupercite{wiki:godelnum}来
形式化,后者将不同的程序映射到不同的整数,而这种操作也可以应用到证明上,因此每个
证明在形式上都等价于一个程序。但这种关系不能反过来,因为证明必须在有限步内结束,
而程序则不必这样;不过因为算法也须要在有限步内结束,每个算法在形式上都等价于
一个证明。}:证明(类似于程序)毕竟是基于公理(类似于基本操作),因此通过将
所涉及定理的证明(类似于库函数的实现)嵌入的方式,我们可以将一个证明归约
为对公理的一系列应用,于是就能计数了。考虑到这种对应关系,我们可以像
重构程序一样\stress{“重构”证明和命题}来让它们更加简短、清晰,例如
\begin{quoting}
	对矩约束下的离散概率测度,具有 (*) 式形式的概率测度只要存在,则必唯一且
	必为最大熵分布;而最大熵分布只要存在,则必然有 (*) 式的形式,因而也必唯一。
\end{quoting}
可以重构为
\begin{quoting}
	对矩约束下的离散概率测度,存在最大熵分布当且仅当存在
	具有 (*) 式形式的概率测度,此时两者必等同且唯一。
\end{quoting}

沿用这一思路,我们应该可以构造 Hilbert 所想的对简洁性的形式化标准,但关于论证
证明最简性的那一部分怎么办呢?我的猜想是应该存在无法论证为最简的证明,这里我用
将在第 \ref{sec:kolmogorov} 节进一步讨论的 Kolmogorov 复杂度来说明这一猜想。
为了论证一个证明的最简性,我们很可能须要寻找可行证明的复杂度下界,这一下界似乎
和 Kolmogorov 复杂度紧密相连;然而即使忽略后者的不可计算性,Chaitin 不完全定理
也证明存在一个复杂度上界,我们无法论证一个证明的复杂度不低于此上界。那么如果
被证明定理的复杂度已经超过这一上界,我们该怎么办呢?此时连可行证明的
最低复杂度都无法论证,这就 Hilbert 所追求的方法而言显然不妙。

\section{Boltzmann 公式:公理还是定理?}\label{sec:boltzmann}

当年在我参加我后来被录取那家单位的考研面试时,一位面试老师问了我“Boltzmann
公式 $S = k_\mathrm{B} \ln\varOmega$ 在统计物理中是公理还是定理?”的问题。
从当时各位老师的反应来看,我的回答“都可以”让他们颇为惊讶,于是我解释了我
想表达的意思是一组命题可以基于多组可以互相替代的基础公理,而一个命题
可以在一些基础之上是公理,同时在另一些基础之上又是定理。也许是这一
完整回答仍然出乎那些老师(主要是物理学家)预料的缘故,他们没有再
追问;但上述问题显然不应到此为止,而我将在本节深入考察这一问题。

在我的回答之后可以立刻追问的一个问题是“那么哪种方案好?”,因为\stress{各组
基础公理并不一样好},尽管它们可以“rebase”到彼此之上:例如我们可以在一组公理中
加入一个已知的定理,由此产生的公理组显然和之前的公理组具有相同的相容性,只是
前者有额外的冗余性。此外,即使在多个互相等价的极小公理组中,有些也可能不便
使用\footnote{出于对函数式编程的好奇,我在还是博士生的时候参加了一次具有很强
科普性质的关于范畴论的报告,其中报告人提到了一些数学家把数学从集合论 rebase
到范畴论的努力。尽管这种努力在学术上可能很有趣,我对其在实际应用中的重要性感到
怀疑:从我极其粗浅的了解来看,对基本数学概念的范畴论形式化似乎比其集合论类似物
要更加复杂,因此根据本节的模型来看,前者似乎并不比后者优越。}:打个比方,尽管
存在许多 Turing 等价的计算模型,计算机科学家在一般性的讨论中最常用的仍然只是
Turing 机、lambda 演算以及其它少数几种模型,而不是元胞自动机乃至 \verb|sed|。

我用一种我称为“\stress{推导图}”的基于图的模型来分析后一问题,这一模型以命题为
节点、以推导为边,因此它和依赖图很相关但又有明显的区别:第一,经常有“等价”的可以
互相推导的命题,所以推导图中一般有环,而非像依赖图那样无环;第二,一个命题常常
可以用多种方式证明,所以推导图中须要有析取节点(也就是“虚节点”);第三,一些
证明比其它的证明更复杂,所以推导图中的边是带权的。有了这一模型,我们就可以
通过从各公理组张成全图时的总路径权重来比较这些公理组,而这种总权重
正好和生成所有命题过程的复杂程度这一直观概念相联系。

推导图在实际的定量应用中用处多半不大,主要原因是此时涉及的形式化系统是无穷的,
但推导图作为一个概念仍然具有很强的指导意义,而我们将在本部分中多次涉及这一概念。
例如推导图有助于解释第 \ref{sec:user} 节中的知识重要性判据:通过掌握具有较强
普适性(联系到图中其它许多部分)和有用性(简化复杂的推导)的概念,我们将能明显
更高效地理解所学的内容。在结束本节之前,我们先暂时回到 Boltamann 公式的问题:
我猜测这一公式在几乎所有合理的极小公理组之上应该都会是定理,只要熵 $S$
在统计物理中还是一个导出概念而非原始概念。

\section{Kolmogorov 复杂度和信息组织}\label{sec:kolmogorov}

在之前各节中,我们已经多次初步涉及在特定场景中复杂度下界的问题:例如在
考虑如何论证给定公理的某一证明的简洁性(见第 \ref{sec:hilbert} 节)时,我
考察了可行证明最小复杂度的概念;早在讨论内聚(见第 \ref{sec:coupling} 节)
时,我把它定义成子模块间本质性的耦合,这也和某种不可避免的复杂度相关联。
在考虑客体的本质复杂度时,人们经常使用 \stress{Kolmogorov 复杂度}%
\cupercite{wiki:kolmogorov},而我们将在本节对这一概念进行进一步考察。

从本质上看,一个客体(通常编码为字符串)的 Kolmogorov 复杂度是可以产生它的
计算机程序的最小长度,其值对所用程序语言只有很弱的依赖性:两个语言所对应
复杂度的值之差具有一个只和这对语言相关、和字符串无关的上界。Kolmogorov
复杂度是一个\stress{不可计算}的函数,换言之我们可以证明根本无法编写一个
能对所有输入计算其值的程序;此外 Chaitin 不完全定理表明可证明的复杂度
下界值存在一个上界,这一上界只取决于证明所使用的语言和证明所用的
公理系统,所以我们甚至无法证明一个字符串比这一上界更复杂。

你一定会好奇,既然 Kolmogorov 有这么“不好”的性质,那它还有多大的意义?据我所知,
Kolmogorov 的确极少用于对复杂度的实际度量,而主要是用来证明各种不可能性,例如
特定函数的不可计算性。然而这也意味着一件事:从上文很容易注意到信息压缩的极限
正是 Kolmogorov 复杂度,因此这一复杂度的不可计算性及其下界在多数情况下的不可
证明性说明对信息压缩的研究将永远不会有一个正式的终结。类似的结论在其它研究领域
中也存在,它们被总结为“full employment theorem\cupercite{wiki:employ}”;在一个
有些哲学意味的层面上看,虽然那些“不好”的性质表明了这些领域中可用形式化理论的
局限性,但是它们同时也表明\stress{人的努力并不会轻易地被计算机程序取代}。

有必要注意,Kolmogorov 复杂度所度量的是已知客体的本质复杂度,而不能轻易扩展到对
多个可能但未知的客体,例如一个定理的所有可行证明之间的比较。此外,Kolmogorov
复杂度在和信息压缩紧密相关但不那么形式化的领域,例如图书馆学中的\stress{信息
组织}领域中适用性更差;我认为广义的信息组织正是编程对系统复杂度的极小化在现实
世界中的对应物,所以它必须或多或少地符合极简主义。当然,正如推导图模型一样,
尽管 Kolmogorov 复杂度并不能定量地应用于实际问题(并不完全如此,我们
将在第 \ref{sec:ockham} 节中看到其一些间接应用),它在概念上
仍然具有指导意义,因其本质是\stress{描述复杂度}。

\section{科学技术中的极简主义}\label{sec:science}

到目前为止,本部分中主要讨论的是数学和假想的公理化物理中的极简主义,但极简主义也
可以在其它的科学技术领域,特别是那些\stress{复杂度会带来可观的有形成本}的领域中
观察到,这些成本既包含经济成本也包括脑力成本。就此而言,传统的工程领域是一个不错
的例子,因为其中复杂度的经济成本在技术进步的背景下仍然明显:只要一个工程项目的
预算有些紧张,如果完成同一任务的简单方法没有其它严重的代价,那么它一般会优先于
复杂方法被考虑。例如 Jon Bentley 在《编程珠玑》\footnote{其在《ACM 通讯》
上的同名专栏正是最初讨论词频排序问题(见第 \ref{sec:shell} 节)
的地方。}中写到\cupercite{bentley1999}:
\begin{quoting}
	General Chuck Yeager (the first person to fly faster than
	sound) praised an airplane's engine system with the words
	``simple, few parts, easy to maintain, very strong''.
\end{quoting}

极简主义显然也是理论物理学家的一种追求,例如《人月神话》的
作者 Fred Brooks 在《没有银弹》\cupercite{brooks1987}中写到:
\begin{quoting}
	Einstein repeatedly argued that there must be simplified
	explanations of nature, because God is not capricious or
	arbitrary.  No such faith comforts the software engineer.
\end{quoting}
这一态度让我们回想起 Plan~9 中系统调用的精简化(见第 \ref{sec:plan9} 节),而
类似的态度在理论物理中的其它著名人物,例如 Issac Newton(注意其《自然哲学的数学
原理》)、James Clerk Maxwell(注意其方程组)等等身上也很明显。我们也可以注意到
以优雅推导的形式体现的极简主义,这和对需求的优雅实现相对应:我的一位学力学的朋友
曾经提到,在不少流体动力学教材花费大量(有时一两章)篇幅分析重力波\footnote%
{注意不是广义相对论中的引力波。}的背景之下,Lev Landau 和 Evgeny Lifshitz
的《理论物理学教程》只用长约 10 页的一节就将其解释清楚,
而这 10 页中还有好几页是习题及其解答。

从一小组规则中推出系统性理论的传统当然可以追溯到更早,而人们通常将 Euclid
的《几何原本》看作最早采取这一做法的最重要资料之一;几何学成为西方教育中一个
不可或缺部分的原因几乎无可争议地就是《几何原本》,但此书在代数上有一个问题:其中
所有的代数命题和推导都是以句子而非公式写成的,这或许是因为当时的数字符号系统太
难用。20 世纪,Bourbaki 数学学派\footnote{他们用于表示警告的弯道符号直接启发了
Donald Knuth 使用“\textdbend”。}在用基于集合论的纯公理化方法构建数学基础的过程中
采用了完全相反的做法:他们的著作的确具有极其深远的影响和非常积极的意义,但其内容
风格极为简练、抽象,少有解释性的评注、只有最少量的例子,而且几乎没有图像、完全
不谈实际应用。这虽然并不影响 Bourbaki 著作在逻辑上的正确性,但是也为试图理解
其内容的读者制造了困难;第 \ref{sec:cognitive} 节会深入分析这一问题,
这里我只希望在结束本节之前再额外讨论一个可能值得考虑的问题。

在讨论当前所用数学基础万一被证明不相容可能造成的结果时\cupercite{gaillard2010}%
\footnote{如果你好奇为什么居然会有这样的问题,注意 \stress{Gödel 不完全定理}%
\cupercite{wiki:godelthm}(事实上是其中的第二定理)论证了这些数学基础不能自证其
相容性,所以这一理论问题是有一定根据的。},有人指出多数学者完全可以找个替代的
数学基础,然后继续自己的日常工作;毕竟集合论诞生于 1870 年代,晚于其它许多
领域,而且许多结果的集合论表述不过是用另一种数学语言重新陈述已有结果:例如
V.~I.\ Arnold 在 1963--1964 年不借助公理化的手段向中学生讲授了群论的内容,并
能在半年内讲到一般五次方程根式解不存在性的一个拓扑学证明\cupercite{arnold1998,
alekseev2004}\footnote{顺便提到,Dan Friedman 的好几部著作,例如《The Little
Schemer》和《The Little Prover》,也达成了类似的目标。}。类似的现象在物理学
中也存在,例如量子力学的现象学结论对实际底层原理的相对独立性,以及宏观
热力学结论对底层统计物理原理的相对独立性;这些现象证明了形式化系统和
半形式化系统中\stress{抽象层之间相对分离}的存在性,就像软件系统中的
类似构造一样:从推导图的角度看,通过加入更加原始公理的方式,我们
可以将一个推导图中的公理转换为另一推导图中的定理,而此时虽然会
出现对现有命题的新证明途径,但是整个图的总体结构不会有太大的改变。

\section{科学和非科学理论中的极简主义}\label{sec:ockham}

在本部分中,我已经提到几种理论,它们中的每一种基于一组极小的基本假设,这些
假设在形式化之后便成为公理;在上节,我提到这样的传统和 Euclid 的《几何原本》
有着密切的联系,但并未把这一传统限制在科学技术领域,因为它其实也被许多哲学家
乃至一些神学家贯彻。一个介于这些领域之间的重要例子是 Pierre-Simon Laplace
在被 Napoleon Bonaparte 问起其五卷巨著《天体力学》中为何没有提及上帝时的
回答:“我不需要这一假设”。这一传统被总结为称作 \stress{Ockham 剃刀}%
\cupercite{wiki:ockham}的原则,通常表述为“若无必要,勿增实体”;有必要
指出,Ockham 剃刀是在多个理论都和已有观测吻合且做出同样的预测时用于在
它们之中进行选择的原则,它并不意味着最简的理论在所有情形下都最有可能正确。

现已存在多种利用和概率论相关的形式化手段来从数学上论证 Ockham 剃刀的尝试,例如
关于最小信息长度\cupercite{wiki:mml}(MML)和最小描述长度\cupercite{wiki:mdl}%
(MDL)的理论;MML 和 MDL 都是受到第 \ref{sec:kolmogorov} 节所讨论 Kolmogorov
复杂度启发的产物,但和 Kolmogorov 复杂度不同的是 MML 和 MDL 可以用于统计模型
的实际挑选。MML 和 MDL 方法之间有比较微妙的区别,但两者大体思路相近:以某种
方法表示一个模型,并把它和观测数据被这一模型编码之后的表示连接起来,而我们
可以证明\stress{具有最短联合表示的模型最有可能是正确模型}。由此可见 MML 和
MDL 在数学上和信息压缩有着很强的关联,这使它们和 Kolmogorov 复杂度相联系;
而如果我们将已有观测看作数据,将理论看作模型,那么从上述结果可知能高效解释
观测数据的最简模型就是最好的模型,这的确和 Ockham 剃刀原则吻合。上述结果
也很容易使我们联想到在减少系统总复杂度这一意义上的 Unix 哲学,如果我们
把希望实现的应用看成数据,并把支持这些应用所需接口的实现看成模型的话。

\section{文学和艺术中的极简主义审美}\label{sec:art}

追求简洁在文学和艺术领域也是一种重要的审美观,而本节将主要从文学方面入手,简要
地考察这种审美观。声名显赫的美式英语格式手册《英文写作指南》在 2011 年被《时代》
杂志评为 1923 年以来影响力最大的 100 部英文书籍之一,其作者 William Strunk
Jr.\ 格外强调“\stress{omit needless words!}”,并在书中对写作风格提出了以下的
建议\footnote{董桥将这段 63 个单词的箴言译为 63 个汉字\cupercite{dongqiao1999}:
“铿然有力之文必简洁。一句之中无赘字,一段之中无赘句,犹如丹青无冗枝,机器无
废件。此说不求作者下笔句句精短,摒弃细节,概而述之;但求字字有着落耳。”}:
\begin{quoting}
	Vigorous writing is concise.  A sentence should contain no
	unnecessary words, a paragraph no unnecessary sentences, for the
	same reason that a drawing should have no unnecessary lines and
	a machine no unnecessary parts.  This requires not that the writer
	make all his sentences short, or that he avoid all details and
	treat his subject only in outline, but that every word tell.
\end{quoting}

《英文写作指南》并不只针对文学作品,因此其反映的其实是所有英语写作中的一种
共同审美,而类似的审美至少早在约 5 世纪的中国就已存在。南北朝时期刘勰的
《文心雕龙》是我国首部系统性的文学批评著作,此书也将所有文体的作品都当作文学
作品来分析。李敖在《中国名著精华全集》中把《文心雕龙》全书重点总结成两点:
\begin{quoting}
	一个是反对不切实用的浮靡文风,一个是主张实用的“摛文必在纬军国”之落实文风。
\end{quoting}
而我国《文心雕龙》学会首任秘书长牟世金这样评价此书的基本观点%
\cupercite{moushijin1995}\footnote{同一资料中也指出,《文心雕龙》
是作者对当时浮夸文风所产生反应的产物,而事实上本文档又何尝不是呢?}:
\begin{quoting}
	要有充实的内容和巧丽的形式相结合,这就是文学创作的金科玉律,这就是
	刘勰评论文学的最高准则。这一基本观点,是贯彻于《文心雕龙》全书的。
\end{quoting}

在对文学作品的评价中,“堆砌辞藻”“华而不实”“空洞无物”等等从来就不是褒义词,由此
可见上述审美观在中外文学界的确是得到充分认可的;从反面来看,可以印证这种认可的
还有中西文艺界对使用微小细节表达丰富内涵的描绘手法,例如在绘画中用寥寥数笔表现
其中人物动态或情感手法的普遍赞赏。又例如中国文学提倡“炼字”、讲究“诗眼”,以下诗句
\begin{quoting}
	鸟宿池边树,僧敲月下门。
\end{quoting}
被世人传诵的原因就在于其中“敲”字既营造了敲门声打破夜晚寂静的氛围,又
(相对于“推”字而言)暗示了门外是访问的客人而非归家的主人。我相信类似的
例子在不同的文明中普遍存在,而且其中反映的审美观在很大程度上并不是从
其他文明中传播过来的;正如十进制数在多种文明中的互相独立产生很可能是
因为人类有10 个手指(“digit”)一样,\stress{极简主义审美能在大相径庭的
各种文明中不约而同地产生应该有其深层原因},而我将在下节探讨这一原因。

\section{从认知的视角看待极简主义}\label{sec:cognitive}

在第 \ref{sec:intro} 节,我问到“追求简洁只有审美意义了吗?”,而读到这里时你
应该已经知道并非如此,因为我们有充分的现实理由去追求审美。在第 \ref{sec:art}
节,我提到极简主义审美在多种文明中互相独立产生应当归结于某种深层原因;事实上,
我认为这一审美有其认知根源,后者和那些现实原因存在紧密的联系,而和具体的文明
无关。在本节中,我将讨论极简主义的认知本源,并且考察其一些表面“例外”;经过这些
分析之后,我们就能进入后两节,也是本部分最后两节的内容。我将本节的论证建立在%
\stress{多数(如果不是所有)道德和审美都源于现实生活中的实际需求}这一基本假设
之上。为什么我们讨厌在有很多人等待时插队的人?因为这种行为牺牲许多人的时间精力
来换取少数几人的利益,而且刺激人们以一种拥挤的方式等待,后者降低总吞吐量且可能
造成安全隐患。为什么设计师讨厌 Comic Sans?因为这种字体在开启抗锯齿的条件下
(这是当今的绝对主流)具有不佳的易读性\cupercite{kadavy2019};这可能有些微妙,
但如果考虑到在你初学写字时老师几乎会本能地制止潦草的书写,这一理由会显得更自然。

Paul Graham 提到,数学家和好的程序员在工作时会将整个问题置于脑海中,以充分探索其
涉及的各方面,因此简洁性在求解问题的过程中自然有其重要性\cupercite{graham2007};
他也提到,在通常的办公室环境中工作的普通程序员很少进入状态,将其正在处理的程序
充分载入脑中。考虑到后一类程序员仍然可以生产质量平庸的程序,这些程序或多或少达到
指定的目标,但数学家却常常无福享受这种宽松待遇,这或许能为第 \ref{sec:boltzmann}
节中数学家显得比程序员更极简主义的现象提供一种解释。我进一步认为在求解问题时将其
置于脑海中的要求不是数学家和程序员专有的,因为\stress{人不擅长于多任务}:借用
计算机科学中的术语,我们可以说人类思维中“上下文切换”有很大的开销,所以为了避免
上下文切换、达成优良的性能,被求解问题的所有相关子问题都须要载入脑中。有了简洁
但强大的工具,我们就能让许多子问题更好理解,从而提升我们能载入脑中的问题规模,
而我相信这就是我们欣赏这些工具的认知根源。 V.~I.\ Arnold 这样回忆他在被教到
看起来无关的数学概念之间存在的本质关联时的本能反应\cupercite{arnold1998}:
\begin{quoting}
	\emph{[... a more impressive example, omitted here for the sake of
	brevity ...]} Jacobi noted the most fascinating property of mathematics,
	that in it one and the same function controls both the presentations of
	an integer as a sum of four squares and the real movement of a pendulum.
	These discoveries of connections between heterogeneous mathematical objects
	can be compared with the discovery of the connection between electricity
	and magnetism in physics or with the discovery of the similarity in
	the geology of the east coast of America and the west coast of Africa.
\end{quoting}
类似地,我在了解到 Aboriginal Linux 的起源\cupercite{landley2017}时也震惊了:
\begin{quoting}
	The original motivation of Aboriginal Linux was that back around
	2002 Knoppix was the only Linux distro paying attention to the
	desktop, and Knoppix's 700 megabyte live CD included the same
	100 megabytes of packages Linux From Scratch built, which
	provided a roughly equivalent command line experience as the
	1.7 megabytes tomsrtbt which was based on busybox and uClibc.
\end{quoting}

V.~I.\ Arnold 的评论本意是要和 Bourbaki 学派(见第 \ref{sec:science} 节)的
作风对照,而你多半会好奇,既然简洁性是这么好的属性,为什么未受训练的读者会在
阅读 Bourbaki 式的数学书时感到困难呢?我的答案建立在第 \ref{sec:mcilroy} 节对人
和机器之间区别分析的基础之上,这一分析在深层次上表明了人和机器分别擅长于完成抽象
和直观的任务。此外,为了充分理解所学的内容,人还需要涉及其各个不同方面的反复刺激
(这或许也能帮助解释为什么人类长于直觉),而我猜想这可能是为了更好识别重要知识
(如之前借助推导图讨论的一样)进行自然适应的产物。这些都告诉我们,虽然形式化
系统只需要命题和推导的抽象表示,人还需要评注、例子、图像和实际应用才能充分
理解它们;正因如此,\stress{形式化的抽象和非形式化的图像是互补的,它们
对我们具有同等的重要性}:例如考虑微积分中极限的 $\epsilon$-$\delta$
定义及其非形式化版本,“通过使参数的值充分逼近指定的输入值,映射
的值可以按要求任意逼近极限”,后者顺便也揭示了极限的拓扑学本质。

在本节末尾还有一个问题须要考虑:我在上文中提到,数学家通常无权像程序员那样生产
“差不多过得去”的结论,但这是为什么呢?因为数学家一般须要严格证明其结论,或者至少
在其强烈怀疑后者为真的前提下提出猜想;相比之下,程序员把软件看成工程项目的产品,
并且因其在本质上的复杂性经常愿意牺牲可证明的正确性以换取更低的开发和维护成本%
\footnote{\label{fn:formal}事实上,\stress{形式化验证}在硬件产业中使用相当广泛
(可能是因为对硬件的要求相对固定),但在软件中仍然极少大规模应用。一个著名的
例子是 seL4 微内核\cupercite{wiki:sel4}的形式化证明,它比 seL4 本身的代码大一个
数量级,而后者本身已经很小而且拥有良好的设计。}。因此在形式化证明成本过高时,
我们不得不依赖我们自己的推理,但正如 Tony Hoare 所说\cupercite{hoare1981}:
\begin{quoting}
	There are two ways of constructing a software design: one way is to make
	it so simple that there are obviously no deficiencies, and the other way
	is to make it so complicated that there are no obvious deficiencies.
\end{quoting}
由此可知尽管机器只用知道程序所对应的可执行指令,人还需要设法保证\stress{程序的
正确性};不管保证正确性的手段是形式化证明还是人工推理,对程序内部机理的清晰理解
都是必需的。所以在评估软件系统时,我们不仅须要度量其尺寸,而且须要考察系统中的
内聚和耦合,而这正是我在第 \ref{sec:complex} 节中强调“认知复杂度”一词的原因。

\section{Unix 哲学在社会中推广的限度}\label{sec:society}

正如我们在本部分已经看到的,除了编程中的 Unix 哲学之外,极简主义在科学技术、
哲学和文学艺术中也有其重要性;这并非偶然,因为极简主义根植于人类认知之中,
这本身很可能是源于人类进行多任务时的巨大开销。由此而来的一个自然的问题是
“Unix 哲学是否能应用于社会?”,因为社会在许多方面类似于机器,因而或多或少地
可以从 Unix 的角度分析。把社会看成机器的一个重要理由是因为分工使社会可以
在概念上被划分为互相关联的群体,其中每个群体执行一组不同的任务,而每个群体
又可以被分为执行不同子任务的子群体。沿用这一思路,我们可以粗略地将社会比作
由互相联系的模块组成的系统,于是 Unix 哲学看来可以在其中找到应用。

考虑到这一背景,设法促成不同社会群体之间(以及类似地在同一群体之内的子群体之间)
的\stress{适度分工}当然是有利的;而极端细化的社会分工似乎也成为一种有些诱人的
设想,在这样的分工下每个人只做一件极其微小且良定义的工作。然而正如 Charlie
Chaplin 的杰作《摩登时代》中生动表现的一样,让人们像没有思维的机器一样工作其实
是违反人类本性的,这样做对社会的坏处远大于好处。我相信这是社会为何常常不像
Unix 那样工作的一个本质原因,它可以用上节中讨论的人类对多样刺激的需求来解释:
如果你整天做的都是某种机械、单调的工作,你满脑子想的都会是这种工作,这会让你的
思维越来越疏远于其它人类行为。由此可见,虽然生产极简主义的产品是好事,但是也
有必要\stress{尊重人的本性},鼓励人们经常尝试一点在思维上有挑战性的不同的
东西;从另一方面来看,多样的刺激也能增强人的创造力,并由此增强整个社会的活力。

正如之前强调的,人类并不能完全比作机器,因为人需要多样的刺激;类似地,社会群体也
不能完全比作计算机系统中的模块,其中一个主要的原因是\stress{社会群体一旦形成就会
追逐自身的利益},而不像软硬件模块那样只是机械地工作。例如正像 Laurent Bercot 等
人注意到\cupercite{ska:systemd}的一样,公司会自然地倾向于生产质量平庸的软件,而
Paul Graham 将其归结于\cupercite{graham2007}公司希望每个开发者是可替代的;这一
趋势(即使只是一种无意识的效果)在依赖贩卖服务以获利的开源公司中注定更为严重,
因为生产难以独立维护的软件更加有利可图,而我强烈怀疑这是催生 systemd 闹剧(见
第 \ref{sec:foss} 节)的因素之一。由此可知,社会群体之间的交互并不完全是合作性
的,因此正如 Unix 式高安全系统\cupercite{djb:qmailsec}中的模块那样,社会群体互相
之间的信任应该有一定的限度,所以在这一方面 Unix 和社会达成了某种殊途同归的效果。

\section{极简主义对个人工作、生活的启发}\label{sec:worklife}

我在第 \ref{sec:user} 节中提到高效学习 Unix 的本质,以及如何实现这种高效学习;
在读完本部分中前面各节之后,你应该已经知道这些并不是 Unix 独有,而是所有学习
过程中共通的原理。既然如此,Unix 在学习过程中扮演的真正角色是什么呢?我认为
Unix 哲学会自然地促成极简主义的习惯,后者正是高效学习的关键;在本节中,我将
讨论如何更好地培养这种习惯,以及这种习惯会为并不出众的人带来怎样的实际收益。

正如之前所述,高效学习的实质在于降低学习过程的总复杂度,而我们对此已经有了一个
概念模型——推导图(见第 \ref{sec:boltzmann} 节);利用这一模型,我们分析了第
\ref{sec:user} 节中的知识重要性判据,而我在这里要对这一分析进行一些补充。我们
知道和死记硬背\footnote{有时我们不得不死记硬背,此时我们可以考虑使用抽认卡程序,
后者常常可以极大提升记忆效率。Mnemosyne 和 Anki 是两个开源的抽认卡程序,但遗憾
的是两者都比较臃肿。}相比,记忆重点并根据需求从它们推出其它知识点通常是更好的
选择;这意味着我们可以把推导图“压缩”成那些普适且有用的概念,并从这些概念游走到
需要的部分。然而必须强调的是\stress{我们记住的不仅是重要概念},而且有对它们和
其它概念之间关系的粗略印象,而这种印象通常是(如果不总是)基于先有知识的:例如在
学习乘法表时,掌握加减法会很有用,因为后者帮助学习者将不熟悉的结果关联到熟悉的
结果(如 $7 \times 6 = 7 \times 7 - 7 = 49 - 7 = 42$)。因此为了能高效学习,上述
印象最好是基于某种扎根于我们脑海中的知识,而我相信后者可以简单地称为直觉:如第
\ref{sec:cognitive} 节中分析的一样,为了充分理解某种客体,我们需要多样的刺激,
而直觉(例如自然数及其基本运算)正是先有刺激的产物,它可以帮助我们降低学习所需
刺激的量;这也支持了同一节中非形式化图像和形式化抽象一样重要的论断。

由上可知为了高效学习,我们不但要关注重要的概念,而且要尝试寻找它们和我们已经
熟悉的概念之间的联系;就后者而言我们可以从主动\stress{寻找概念之间的相似性}%
开始,例如“\verb|fork()|/\verb|exec()| 像原型模式”“12 个音符间的音程就像钟表上
12 个小时之间的时差”。通过这样的做法,我们能使用先有知识来极小化我们须要记忆的
信息量,从而实现所谓“把书读薄了”的效果;此外,在发现概念之间的新联系时,对旧
概念的理解也会被加深,而正如第 \ref{sec:devel} 节所述,对相关主题的深刻理解
往往是优雅发明的关键。从概念模型的层面上看,在发现不同推导图在结构上的相似性
之后,每个图中的推导经常可以为发现其它图中之前被忽视的推导提供无价的启发;
在这一方面,跳动的宏观液滴和被量子力学支配的微观粒子之间的类比,以及我国
钟万勰院士关于经典动力学和结构力学在计算方面类比的工作是两个很好的例子。

于是我们已经考察了多个推导图之间的关联,并考虑了先有知识可以怎样帮助我们
掌握新概念;正如上文所述,\stress{新知识也有助于我们从新的角度考虑先有概念}:
换言之,“温故而知新”,这正是我们应该培养回顾旧知识习惯的原因,同时也是我在第
\ref{sec:devel} 节中提到我们应该经常考虑重构的原因。既然再次提到了习惯的问题,
我认为现在是时候指出直觉和习惯都是重复刺激的产物,它们的区别在于前者主要关注
“某物是什么”(例如柠檬是酸的),而后者主要关注“怎样做某事”(例如怎样使用某种
餐具)。\stress{好的直觉会给人很丰厚的回报,但它经常需要一定量的训练才能获得}:
例如一个在有机化学上经过足够训练的人可以立刻察觉 \parencite{zxhxy2018}\footnote%
{这显然不是其原始来源,但我的确没能找到原始来源;据我所知,这一合成路线至少可以
追溯到 2014 年,当时我在看到这一路线之后把它的最后一步画在了一件 T 恤衫上。}%
中囧烷(经过环猫烯)合成路线的优雅,但这种训练即使对在最优秀老师指导之下的
最聪明学生也并非易事。类似的结论对习惯也成立,这也为我们提了个醒:如第
\ref{sec:society} 节所述,我们应当尊重人的本性,但同时纵容自己保持从长远
上看弊大于利的习惯也是不明智的;打个比方,虽然慵懒的坐姿可能感觉更舒服,
但是它一般会导致近视、驼背和鸡胸,而我可以证实这三者都是不健康的。

在本部分的末尾,我希望在给出最后的观点之前考虑一个古典音乐的例子:“多数
音乐家同意 J.~S.\ Bach 是西方音乐史上最伟大的作曲家,而 L.\ van Beethoven
和 W.~A.\ Mozart 争夺第二的位置”这种说法应该不会显得过于不公,然而同时几乎没有
人会对 Mozart 的天赋明显高于 Bach 这一点有争议,这是为什么呢\footnote{类似的比较
或许也可以在 Albert Einstein 和 John von Neumann 之间进行,虽然对两者伟大程度的
判断将比对 Bach 和 Mozart 的判断更有争议得多。}?我认为答案已经蕴含在我一位同学
注意到的现象之中:“Mozart 总有新的乐思,而 Bach 不断探索已有乐思中新的可能”;
或许证实因为这一原因,Bach 作品中的元素经常被称为许多后世音乐风格的原型,这些
风格最晚在 20 世纪才产生。我想从这个例子中表达的观点是\stress{尽管“天才”可以
轻松地深入思考,他们也往往缺乏努力思考和回顾旧想法的动机},因为有太多足够有意义
的问题供他们求解,但这同时也留下一些重要但未能考察的的死角。在编程中,这些死角
中一个极为重要的部分是软件系统的简化,这一工作即使普通程序员也常常能做,只要
后者有足够的决心;在实践中,这种简化对我们普通程序员的意义反而更大,所以我认为
我们应该比“天才”更有动力去做这一工作。此外如上文所述,极简主义的习惯让我们专注
于问题的本质,这会自然地引起深入思考,并因此常常可以提供深刻的眼光,后者正是
解决难题的关键。有时伟大的程序员和我们之间的区别正是这样的眼光,这或许也为第
\ref{sec:user} 节中 Dennis Ritchie 的名言提供了另一种解释——通过贯彻极简主义,
即使并不出众的程序员也可以成为“天才”;出于这一原因,我以下面的引言结束本部分:
\begin{quoting}
	Common sense is instinct, and enough of it is genius.
\end{quoting}


\newpart
\section{从 Lisp 到 Scheme}\label{sec:lisp}

在上一部分关于编程之外极简主义的讨论之后,现在我们回到编程的主题;但是在考察
Unix 本身之前,我们先来讨论 Unix 之外的一样东西——Lisp 族的程序语言\cupercite%
{wiki:lisp}。诞生于 1958 年的 Lisp 和 Fortran(1957 年)、Algol(1958 年)和
Cobol(1959 年)是最古老的几种程序语言,而 Lisp 或许是其中唯一在当今仍能不断吸引
新人关注的:Fortran 在科技数值计算之外少有应用(而且仍在减少),极少有年轻程序员
对维护现有 Cobol 代码(陈旧且也在减少)感兴趣,而说作为一种语言的 Algol 早已死亡
并无不公。那么 Lisp 能保持其生命力的原因是什么呢?答案在于其天然的简洁性以及并不
因简洁而打折扣的强大\stress{表达力},而我们就将在本节简要考察 Lisp 的简洁性。

John McCarthy\footnote{另外,他也是分时操作系统的先驱之一。} 设计了 Lisp,并由此
说明了我们从几个基本函数、条件表达式和递归函数定义就可以建立一种 Turing 完全的
语言\cupercite{mccarthy1960}。Steve Russell 出乎 McCarthy 预料地注意到 Lisp
中的 \verb|eval| 函数可以直接翻译为汇编代码,其结果就是第一个 Lisp 解释器,
后者运行在一台 IBM 704 大型机上、由 Russell 人工编译而成。Smalltalk 的
设计者 Alan Kay 就这一想法发表了以下的评论\cupercite{feldman2004}:
\begin{quoting}
	\emph{[...]} that was the big revelation to me when I was in graduate
	school -- when I finally understood that the half page of code on the
	bottom of page 13 of the Lisp 1.5 manual\cupercite{mccarthy1962} was Lisp
	in itself.  These were ``Maxwell's Equations of Software!''  This is the
	whole world of programming in a few lines that I can put my hand over.
\end{quoting}
Lisp 可以被约化为一个极小的核心,但非所有 Lisp 族语言都追求极简:例如现在
“Lisp”一词在不加限制时指的经常是 Common Lisp,后者即使就核心功能而言也比较大;
但 Common Lisp 专家,如 Paul Graham\cupercite{graham1993},仍然明确鼓励通过合理
抽象来压缩具有高重复性的代码。然而真正极简主义的 Lisp 的确存在,那就是 Scheme。

根据 Scheme 的设计者 Gerald Jay Sussman 和 Guy L.\ Steele Jr.\ 的说法,%
Scheme 并不是在设计之初就明确了极简主义的目标\cupercite{sussman1998}(注意这
和第 \ref{sec:mcilroy} 节所述的 Unix 在管道出现之后对极简主义的追求何其相似):
\begin{quoting}
	We were actually trying to build something complicated
	and discovered, serendipitously, that we had accidentally
	designed something that met all our goals but was much simpler
	than we had intended.  \emph{[...]} we realized that the
	\stress{$\bm{\lambda}$-calculus} -- a small, simple formalism --
	could serve as the core of a powerful and expressive programming language.
\end{quoting}
无论如何,极简主义成为了 Scheme 的一个标志属性,这可以从其标准《Revised$^3$
Report on the Algorithmic Language Scheme》及其后续版本(一般简称为 \rnrs{3}、%
\rnrs{4} 等等)\cupercite{scmreports:home}\footnote{\label{fn:r6rs}\rnrs{5} 及
更早版本的标准都只有几十(一般不到 60)页,其中还包含了示例和设计依据;而为了使
一些(多少合理)需要的在主流语言中存在的特性标准化,\rnrs{6} 明显地扩大了语言
核心和标准库的尺寸\cupercite{weinholt2019}。对 \rnrs{6} 的争议很大,其主要原因
是特定的特性对包含教师和业余爱好者在内的那些只使用一小部分功能的人不那么有用%
\cupercite{scmreports:position};为此\rnrs{7} 被特意分为一个“小”标准和一个
“大”标准。顺便提到,Go 的标准也是约 50 页长,但 Go 的表达力远不如 Scheme%
(一个例子见第 \ref{sec:homoiconic} 节)。}中引言部分的第一句话看出来:
\begin{quoting}
	Programming languages should be designed not by piling
	feature on top of feature, but by removing the weaknesses and
	restrictions that make additional features appear necessary.
\end{quoting}
这一态度明显和 Unix 哲学相似,它使得 Scheme 成为在我看来表达力最强
的语言,因此在本文档的后面部分我都使用 Scheme 作为 Lisp 的代表。

本节我们离题去讨论 Lisp 的原因在于其不因简洁性而受损的强表达力,而这一表达力
在很大程度上来源于 Lisp 代码的写法——\stress{S 表达式}(“符号表达式”的简称)%
\footnote{在最初那篇关于 Lisp 的论文\cupercite{mccarthy1960}中,McCarthy 希望
源代码使用的是另一种称为 M 表达式(“元表达式”的简称)的格式,而机器会将其翻译
为 S 表达式;然而程序员多数更喜欢直接使用 S 表达式。M 表达式和 S 表达式一样
具有同像性,而一个和前者紧密相关的格式被计算软件 Mathematica 使用。};例如
阶乘函数在 Scheme 和 Python 中分别可以写成\footnote{它们并不完全等价:
首先,Scheme 没有 \texttt{return} 原语,而是使用\stress{延续}(见脚注
\ref{fn:cps})以及\stress{尾位置}表达式来返回值;第二,在 Scheme 中
\texttt{(define (fn (args ...)) ...)} 是 \texttt{(define fn (lambda
(args ...) ...))} 的简写,而 Python 中的 \texttt{lambda} 在使用上
有严重的限制(一个例子见第 \ref{sec:homoiconic} 节)。}:
\colskipa\begin{multicols}{2}
\begin{quoting}
\begin{Verbatim}
(define (f i)
  (if (> i 1)
      (* (f (- i 1)) i)
      1))
\end{Verbatim}
\end{quoting}
\begin{quoting}
\begin{Verbatim}
def f(i):
  if i > 1:
    return i * f(i - 1)
  return 1
\end{Verbatim}
\end{quoting}
\end{multicols}\colskipb\noindent%
S 表达式对新手通常会显得不直观,但它被计算机处理起来其实比其它多数替代格式
容易地多,而且在稍加适应之后对程序员仍然比较可读。你可能会注意到这和文本流
成为传统 Unix 工具所用接口的原因(见第 \ref{sec:mcilroy} 节)非常相似,
那么为什么 S 表达式对机器更友好,它能为我们带来怎样的技术优势?请看下节。

\newpart
\section{S 表达式和同像性}\label{sec:homoiconic}

为了理解 S 表达式对机器的友好性,我们首先须要了解程序源代码在计算机内部
的表示方式:不管你使用什么语言,源代码都会被转换(\stress{语法解析})
为一种称作\stress{抽象语法树}\cupercite{wiki:ast}(AST)的数据结构,
这种数据结构用于表示计算机在理解程序时所需的语法元素。例如和表达式
$4 \times 2 - (a + 3)$ 所对应的 AST 和 S 表达式分别如下所示:
\colskipa\begin{multicols}{2}
\begin{quoting}[innerleftmargin = 0.45em,
	innertopmargin = -0.15em, innerbottommargin = 0.15em]
\begin{forest}
	for tree = {l sep = 0pt, l = 2.25em}
	[$-$,
		[$\times$, [$4$] [$2$]]
		[$+$, [$a$] [$3$]]]
\end{forest}
\end{quoting}
\columnbreak\vspace*{-0.81em}
\begin{quoting}
\begin{Verbatim}
(-
  (* 4 2)
  (+ a 3))
\end{Verbatim}
\end{quoting}
\end{multicols}\colskipb\noindent%
S 表达式可以看作嵌套列表的表示,而我们从上面的例子可见嵌套列表自然地映射到
包含 AST 的树状结构;此外,虽然 S 表达式并没有完全消除语法解析的需求,但是
它们从技术上看要比其它格式要容易解析得多,这使它们成为一种最方便的表示 AST 的
方式。因此在和 Lisp(原意为“LISt Processor”)强大的列表操作功能结合起来之后,%
S 表达式使得 Lisp 可以几乎不费吹灰之力地处理 Lisp 代码,就像处理普通数据一样。

当一种语言可以像修改其它数据一样修改用其自身写成的源代码时,它就被称为一种
具有\stress{同像性}\cupercite{wiki:homoiconic}的语言;将代码看作数据的能力
有着极为深刻的意义,这从半搞笑文档《Tux 福音》的开头中可见,它正是我把同像性
这一概念看作和复杂度一样深刻的原因\footnote{除了 von Neumann 体系\cupercite%
{wiki:neumann}之外,“代码即数据”这一思想的深刻性还可以被追溯到更早,例如
Gödel 编码\cupercite{wiki:godelnum}和停机问题\cupercite{wiki:halt}。}:
\begin{quoting}
	In the beginning Turing created the Machine.  And the Machine was crufty
	and bogacious, existing in theory only.  And von Neumann looked upon
	the Machine, and saw that it was crufty.  He divided the Machine into
	two Abstractions, \stress{the Data and the Code, and yet the two were
	one Architecture}.  This is a great Mystery, and the beginning of wisdom.
\end{quoting}
第 \ref{sec:boltzmann} 节提到所有 Turing 等价的计算模型并非一样好用,而
类似地所有同像语言在“代码及数据”的实际应用中也并非同样好用\footnote{另一个
例子是 XML,它是 SGML 的后继者,而 SGML 是一种文档标记语言,这使 XML 的
标记部分比较冗长。尽管 XML 自然地代表一种树桩结构,它在用于文档标记之外时
经常产生包含标记多于实际内容的文件;此外 XML 的复杂标准使其不易进行语法解析,
因此其不适合用作程序的源代码。}:例如我们或许可以说所有支持文本处理的语言
在某种意义上都具有同像性,因为源代码毕竟是文本;然而在只用文本处理机制
时,实现解释器和编译器的人所关心的多数\stress{代码变换}实现起来很麻烦。
因此在本文档的后面部分中,在讨论同像性时,我都专指 \stress{AST 级}
的同像性,也就是能像修改普通数据一样轻松修改源代码 AST 的能力。

在具有 AST 级同像性的语言中,宏系统可以实现成使用变换 AST 的函数作为宏;
利用这种宏,我们可以轻松地定义新语法,于是许多硬编码在非同像语言中的%
\stress{语言特性}可以实现为对核心语言的优雅扩展。例如一般被当作 Go
语言主要特点之一的 goroutine 本质上是 libthread\cupercite{lucent2002}/%
libtask\cupercite{swtch:libtask} 模式轻量级协程\footnote{\label{fn:cps}%
另一种模式是基于\stress{continuation-passing style}\cupercite{wiki:cps}%
(CPS)的。}所提供接口的\stress{语法糖},它们被硬编码在 Go 中的原因正是
Go 不能轻松地添加新语法。这个关于 goroutine 的例子可能稍显抽象,因为我们
没有看到实际工作中的代码变换,于是我在这里给出一个更具体且更高级的例子。

匿名函数本质上就是 $\lambda$ 表达式,它在作
函数参数时很有用,正如下列 Python 代码所示:
\begin{quoting}
\begin{Verbatim}
sorted(l, key = lambda e: e[0])
\end{Verbatim}
\end{quoting}
然而 Python 中 \verb|lambda| 的函数体必须是一个表达式,于是我们不能用赋值
语句,例如下面左侧的函数不能直接变换为一个 \verb|lambda|;然而如果我们把
赋值看成将一组输入值(如 $(x, y)$)变为一组更改后输入值(如 $(x, y')$)
的过滤器,这一函数可以变换为多个 \verb|lambda| 的复合,正如右侧所示:
\colskipa\begin{multicols}{2}
\begin{quoting}
\begin{Verbatim}
def f1(x, y):
  y = y / (1 - x)
  x = x + y
  return x * x + y
\end{Verbatim}
\end{quoting}
\begin{quoting}
\begin{Verbatim}
f2 = lambda x, y:
  ((lambda v: v[0] * v[0] + v[1])
  ((lambda v: (v[0] + v[1], v[1]))
  ((x, y / (1 - x)))))
\end{Verbatim}
\end{quoting}
\end{multicols}\colskipb\noindent%
\verb|f2| 的定义明显很丑陋,这种丑陋部分源于 \verb|f2| 中直接写出的值元组
\verb|v|,另一部分源于过滤器顺序的反转——最后应用的 \verb|lambda| 被写在最前,
因为它是最外层的函数。在 Scheme 中,\verb|let| 表达式的输入值写在其主体的前面,
因此我们可以利用它来把过滤器以我们希望的顺序进行复合,正如下面左侧所示。
\colskipa\begin{multicols}{2}
\begin{quoting}
\begin{Verbatim}
(define f3
  (lambda (x y)
    (let ((y (/ y (- 1 x))))
    (let ((x (+ x y)))
    (+ (* x x) y)))))
\end{Verbatim}
\end{quoting}
\begin{quoting}
\begin{Verbatim}
f3 = lambda x, y: \
  (lambda y:
    (lambda x: x * x + y)
    (x + y)) \
  (y / (1 - x))
\end{Verbatim}
\end{quoting}
\end{multicols}\colskipb\noindent%
正如你或许已经猜到的,\verb|let| 只是 \verb|lambda| 的语法糖,\texttt{(let
((a \emph{e$_1$}) (b \emph{e$_2$}) ...) ...)} 是 \texttt{((lambda (a b ...)
...) \emph{e$_1$} \emph{e$_2$} ...)} 的简写,因此 \verb|f3| 等价于上面右侧的
Python 函数。你应该也已经注意到 \verb|v| 未在 \verb|f3| 中出现,这能正常工作
的原因是内层 \verb|lambda| 的参数(如 \texttt{(lambda x: x * x + y)} 中的
\verb|x|)遮盖了外层的同名变量(如 \verb|f3| 的参数 \verb|x|)。

\colskipa\begin{multicols}{2}
\begin{quoting}
\begin{Verbatim}
f4 = lambda x, y: check(
  divide(y, 1 - x),
  (lambda y:
    (lambda x: x * x + y)
    (x + y)))
\end{Verbatim}
\end{quoting}
\begin{quoting}
\begin{Verbatim}
check = lambda arg, fn: \
  None if arg == None else fn(arg)

divide = lambda a, b: \
  None if b == 0 else a / b
\end{Verbatim}
\end{quoting}
\end{multicols}\colskipb
这一例子并没有在上面结束,因为 $1 - x$ 可能为零,而我希望函数在这种
情形下返回 \verb|None|;但考虑到函数只在应用到具体参数时求值,我们可以将
\verb|f3| 变换为上面左侧的函数,其中涉及的辅助函数在右侧。此外,我没有
告诉你 $x$ 和 $y$ 可能是 \verb|None| 而非数值,但我们可以沿用上述思路,
将 \verb|f4| 变换为\footnote{顺便提到,这令人联想到 CPS(见脚注
\ref{fn:cps}),而这种关联并非巧合\cupercite{troelskn2009}。}
\begin{quoting}
\begin{Verbatim}
f5 = lambda x, y: check(
  x, (lambda x: check(
    y, (lambda y: check(
      divide(y, 1 - x), (lambda y: check(
        x + y, (lambda x: x * x + y))))))))
\end{Verbatim}
\end{quoting}
利用这一技术,我们其实是纯函数式地实现了异常\footnote{反过来看,我们也
可以发现异常事实上有类型论的基础。此外我觉得有必要指出 Lisp 不像 Haskell
那样追求纯函数式编程,虽然函数式的写法肯定是优先使用的。},而如果你熟悉
Haskell,你可能意识到这本质上是在使用 monad。\verb|f5| 的一个 Haskell
对应物可以写成下面左侧的形式,后者等价于右侧使用 Haskell 中硬编码语法糖的
形式\footnote{\texttt{return} 在 Haskell 中只是一个函数,而右侧的两个
\texttt{return} 严格对应于左侧的两个 \texttt{Just},其原因是 Haskell 中的
类型论要求。}。那么 Scheme 和 S 表达式在这里扮演什么角色呢?正如刚刚提到的,
上述语法糖是硬编码在 Haskell 中的,而在 Scheme 中它可以实现为语法扩展。
% XXX: hackish.
\pagebreak
\colskipa\begin{multicols}{2}
\begin{quoting}
\begin{Verbatim}
f5 = \x y ->
  (x >>= \x ->
    (y >>= \y ->
      (divide y (1 - x) >>= \y ->
        (Just (x + y) >>= \x ->
          Just (x * x + y)))))
\end{Verbatim}
\end{quoting}
\begin{quoting}
\begin{Verbatim}
f5 = \x y -> do
  x <- x
  y <- y
  y <- divide y (1 - x)
  x <- return (x + y)
  return (x * x + y)
\end{Verbatim}
\end{quoting}
\end{multicols}\colskipb

在本节末尾,我认为有必要指出操作系统中的\stress{目录树是类似于 S 表达式的},
因为它们本质上都是树状结构;Unix 中“一切皆是文件”的原则(见第 \ref{sec:plan9}
节)正是这一思想的直接结果,尽管对 Unix 做出重大贡献的多数人大概并没有注意到
S 表达式的类似潜力:例如 Daniel J.\ Bernstein 在讨论安全性时指出文本数据的
语法解析和格式化输出明显增加了软件系统的复杂度\cupercite{djb:qmailsec},
于是他大量使用了具有简单格式的目录树来配置他的软件,这种做法也被其他一些
和他的软件有紧密关联的程序员采用,这可以在多数的 daemontools 式软件中看到。

如果我们在处理具有复杂结构的数据时使用的是具有同像性的格式,语法解析的复杂度本来
可以被极大降低;同像性对 Unix 哲学的意义将在本文档的剩余部分中进一步考察,而这里
我想多考虑一点关于目录树的问题:既然目录树和 S 表达式类似,那么为什么我们不直接
把多数的配置目录替换为包含 S 表达式的文件呢?我的观点是 S 表达式文件的确比目录树
更好处理,但它也更加固定和单一;相比之下,在使用目录树时我们可以原子性地替换
配置的一部分,而且可以为不同的部分分配不同的权限。后两点在使用单个文件
时实现起来都要复杂得多,而我相信它们表明了目录树作为配置接口的优势。

\section{New Jersey 作风和 MIT / Stanford 作风}\label{sec:wib}

我们在上节看到了同像性无可匹敌的威力,因为它使宏系统可以轻松地操作 AST,从而
实现新的语言特性;此外,因为宏展开是递归的,程序后面部分定义的宏可以使用前面
部分定义的所有语言特性。在编写高性能解释器和优化编译器时,同像性让我们不但
可以使用这些\stress{结构宏}来紧凑地编写要在被处理的源代码上进行的变换,而且
可以将这些代码变换实现为多个简洁清晰且可靠的步骤,后者就像通过管道连接的传统
Unix 工具一样。这种\stress{多步处理}方案\footnote{\label{fn:slew}注意到第
\ref{sec:homoiconic} 节所提到的目录树和 S 表达式之间的相似性,我们也可以实现
配置目录的多步预处理器\cupercite{gitlab:slewman};进一步地,从原型模式的角度,%
\texttt{fork()}/\texttt{exec()} 和 Bernstein chainloading 也可以看作类似的
设计。}的最佳实例是 nanopass 框架\cupercite{andersen2016}\footnote{顺便提到,%
nanopass 主要作者之一的姓名首字母简写是“AWK”。},它被久负盛名的 Scheme 编译器
Chez Scheme\cupercite{wiki:chez} 使用,后者在 2016 年被开源;Chez Scheme
生成的可执行文件经常比用 C 语言编写的类似物还快,但其代码量却比
GCC 小了不止一个数量级,而且自编译仅耗时数秒(GCC 需要数千秒)。

正如从上文已经可见的,直白地说,C 在很多方面和 Lisp 相比都是更差的语言%
\cupercite{graham2002},而下节会进一步考察两者各自的优缺点;C 的问题是多种历史
因素和方法学因素共同影响的结果,而我这样总结这些因素:因为硬件限制,C 起初基本上
是一种可移植的汇编语言,它比较容易实现而且在 Bell 实验室的计算机上性能可以接受;
因为 Unix 的成功,人们不断努力为 C 制造更强的优化编译器,因此它仍是性能最佳的
语言之一;虽然 C 和其它语言相比有着种种弱点,但是人们通常要么不觉得这些弱点是
主要问题,要么直接使用其它语言。简单地说,C 被 Unix 采用最初是因为前者容易实现
且比较容易使用,后来则是主要因为人们的惯性;前一理由反映了贯穿 Unix 发展过程的
一种实用主义,其一个著名的总结是 Richard P.\ Gabriel 的“worse is better”%
\cupercite{dreamsongs:wib}\footnote{正如你从那个页面可见的,Richard
P.\ Gabriel 仍未确定哪种作风更佳。此外 Unix 中可中断系统调用的复杂度
问题可以通过用户空间的包装函数来简单地绕过,后者不断重试直到系统
调用返回且未被中断;然而当前的标准并不要求这样的包装,这是 Daniel
J.\ Bernstein 自己建立一组接口(见第 \ref{sec:devel} 节)的原因之一。}。

Gabriel 比较了 Lisp 和 Unix,其总结大致是在接口的简洁性和实现的简洁性冲突时%
\footnote{Richard P.\ Gabriel 后来在《Worse is Better is Worse》中提到人们
经常忽略这一核心前提,盲目地追求只实现 50\% 需求;我认为忽略这一前提的另一种
表现是在试图完整地实现需求时反对重构,并以“worse is better”为借口\cupercite%
{chiusano2014}。},Lisp 会选择前者(“\stress{the right thing}”,或 MIT /
Stanford 作风),而 Unix 会选择后者(“\stress{worse is better}”,或 New Jersey
作风)。除了 C 以外,“worse is better”在 Unix 中的另一种主要体现是传统 Unix
工具对文本接口的普遍使用\footnote{除了关于纯文本和 S 表达式的争论,其实还有关于
文本格式和二进制格式的争论;在我看来,从 CDB 在 qmail 和 s6-rc 中的使用等等实例
可见,Unix 并不是总避免使用二进制文件,而是只在有简单的使用文本且不影响需求
(如性能和安全性)的方法时才这样做。}(见第 \ref{sec:mcilroy} 节);和 C 类似,
文本接口经常被 Lisp 圈批评,因为纯文本不像 S 表达式之类的同像格式那样有结构。
然而从最小化包含接口和实现的系统总复杂度的角度看,我相信\stress{纯文本
和 S 表达式的选择并不是非黑即白的}:在处理结构简单的数据时,使用纯文本
一般更好,因为这时的接口仍然足够好用;相反在处理结构复杂的数据时,使用
S 表达式缩减的接口复杂度则经常超过增加的实现复杂度。在处理语言时须要
面对的大量数据具有深嵌套且经常自指的结构,因此 S 表达式无疑更有优势。

Lisp 圈另一种常见的对 Unix 的批评是后者把底层细节无情地暴露给用户,而这些
细节本该通过封装的方式对用户隐藏\footnote{从这种批评,我们也可以理解 Lisp
为何更强调接口复杂度。};然而封装时的一个常见问题是\stress{有遗漏的抽象},
此时定制需求常常只能通过绕过封装的方式实现。本文档第 1 部分考察的 systemd
以及来自 Red Hat 的 Linux 发行版都是比较严重的例子,而事实上我认为只有
很少的抽象达到了令人满意的严密度,这些正面例子包括设计优良的程序语言
以及类 Unix 操作系统的内核。即使在区分“普通”和“高级”用户的前提下,在不
干扰后者定制、调试便利的前提下为前者提供封装也并不像是一个已解决的问题。

\section{Unix 和 Lisp 统一将带来的好处}\label{sec:benefits}

如上节所述,C 和 Lisp 相比是一个更差的语言,最主要是因为其缺乏表达力:例如要是
其当初用的是(像 Dale\cupercite{github:dale} 中那样的)可以修改 AST 的宏系统,
那么诸如面向对象编程的一些特性就可以用宏实现,C++ 或许就不再必需。C 的另一个在
Lisp 圈内外广为提到的显著缺陷是其不内存安全,这导致了缓冲区溢出、空指针解引用
等等漏洞;skalibs(见第 \ref{sec:devel} 节)等替代用 C 库当然能帮助减少这类
bug,但远不能把其数量削减到最小,对此我将在下节提到一种可行策略。

然而公平地说,Lisp 也有它的弱点,其中最为大家熟知的是动态类型检查的性能
开销,特别是在数值计算等场景下的开销;自动类型推导,即使是其在 Chez Scheme
等中的高级形式,并不是这一问题的通用解决方案。另一个主要问题是 Lisp 实现所需
的垃圾回收(GC)机制,这使得编译出来的可执行文件更大且经常不适用于实时环境;
此外底层开发中的高级需求,例如经常既需要常数时间运行又需要高性能(往往用到
汇编)的密码学开发(参考 NaCl 和 BearSSL)似乎在很大程度上被 Lisp 圈忽略。
总结起来,根据我了解的信息,我觉得 Lisp 虽然擅长通过抽象实现复杂的应用
需求,但是在底层开发中涉及较少,而它在这一方面本来是有很大潜力的。

上述现象中有一些有趣的东西——Lisp 和 C 的优缺点似乎是互补的,这自然让我们
猜想,如果我们能设法创造\stress{一种吸取两者优势且避免两者劣势的极简主义
同像语言},那么这种语言用 Tony Hoare 的话来说将是\cupercite{hoare1981}%
\footnote{必须从那场演讲中吸取的最重要教训之一,正如 Tony Hoare
所述,是在追求终极语言时必须将简洁性作为头等重要的判据。}
\begin{quoting}
	\emph{[...]} a language to end all languages, designed to meet the needs
	of all computer applications, both commercial and scientific, \emph{[...]}
\end{quoting}
但这能实现吗?如果是,怎样实现,我们能得到什么?如果不是,
这种尝试是否值得?我对第 1 个问题的猜想是肯定的,下一节和第 \ref{sec:toe} 节将
分别探索第 2 和第 4 两个问题,而在本节剩下的部分我将就第 3 个问题给出一些例子。

第 \ref{sec:exec} 节中提到 Laurent Bercot 将 chainloading 的单元操作实现为了一组
命令行工具,而为了实现涉及两个命令的单元操作(例如循环和管道),他设计了一种特殊
的命令行参数转义法,并将其包装为他的称作 execline 的语言\footnote{其设计和 Unix
起初使用的 Thompson shell\cupercite{wiki:thompson} 相似,后者在 Unix v7 中
因为在比较复杂的编程中不够用而被 Bourne shell 取代。}中的语句块\cupercite%
{ska:elblocks}。然而 chainloading 不必然要用 \verb|exec()|,而可以在类 shell
语言中实现\cupercite{vector2016a}(参考 \verb|cd| 和 \verb|umask| 等等 shell
命令);基于 scsh\cupercite{scsh:home} 的模型,execline 语言可以替换为类 scsh
的 Scheme,而 chainloader 可以替换为和下面重新实现 execline 所提供 \verb|cd|
chainloader 的 shell 脚本相似的类 scsh Scheme 程序\cupercite{vector2018c}:
\begin{quoting}
\begin{Verbatim}
#!/bin/rc -e
cd $1; shift; exec $*
\end{Verbatim}
\end{quoting}

沿用这一思路,我们并非无法想象将传统 Unix 工具替换为 Scheme 程序;考虑到
Scheme 的表达力,以及优雅的 Scheme 编译器如 Chez Scheme 的存在,这肯定能
帮助降低系统的总复杂度。所以现在可见,通过迁移到使用 Scheme 进行系统编程,
即使这些简单工具也可以进一步简化;而正如之前指出的,Lisp 在应用编程中
很可能更加强大,因此我确定无疑地认为\stress{通过利用同像性,我们
将能构造在符合 Unix 哲学的程度上超越从前可能的系统}。

既然我们已经看到同像性在系统编程中的威力,现在是时候回顾第 \ref{sec:security}
节中的 Trusting Trust 问题了;当时我指出一种对抗 Trusting Trust 的方法是从机器码
经过多步构造编译器,而应该不言自明的是这一过程的可审计性直接取决于其总复杂度。第
\ref{sec:lisp} 节中又提到 Steve Russell 通过手写汇编实现了第一个 Lisp 解释器,
这和 Chez Scheme 的优雅相结合告诉我们\stress{同像性将极大地降低从机器码构造
编译器全程的总复杂度},因此它能增强我们对 Trusting Trust 的防御\footnote%
{另一个潜在问题是硬件后门,而我认为一个遏制它们的办法是在构造编译器全程中的
每一步预留足够的变通空间,因为底层代码的语义难以分析(这正是开源重要性的来源),
因此任何机械地植入恶意代码的尝试都可能触发假阳性样本,后者可能导致后门暴露。}。

\section{如何统一 Lisp 和 C?}\label{sec:howto}

第 \ref{sec:devel} 节强调我们应当具体问题具体分析,因此尽管资源限制的消失并不
意味着简洁性不再重要,它肯定可以促使我们重新思考之前因为这些限制而作出的妥协。
我们早已度过 GC 在普通计算机上昂贵到无法忍受的年代,而对于资源严重受限的设备我们
一般优先选择交叉编译而非原生编译。基于这一事实,并考虑到上节提到的可执行文件
尺寸以及在实时环境下的需求等等问题,这里提出的语言(我给它取的代号是“Nil”,意为
uNIx 加 Lisp)应当能在解释器和编译器中使用 GC,但\stress{编译器生成的可执行文件
应当在合理的前提之下尽量避免使用 GC}\footnote{然而因为我在第 \ref{sec:benefits}
节提出将 shell 替换为 Nil(当时披着 Scheme 的外衣),一个自然的问题是 Nil 解释器
应该怎样在那样的特殊环境下存在。对此我还没有一个好的答案;或许我们可以使用一个
专门模拟 shell 的缩减版解释器?},而程序员在实现特殊需求时应该清楚怎样避免 GC。

我们在上节回顾了自举的问题,而我暗示了一种方案,其中先设法构造一个极小的解释器,
然后在某个后续步骤中生成一个可用于生产环境的编译器;其中首先须要注意的一点是这
没有采用当今通行的使用编译器来生成解释器的做法,而是首先生成解释器。然而正如
可以猜到的,用汇编写 Nil 解释器多半会比写 Nil 编译器容易,就像 Nil 所模仿的
Lisp 一样;因此从复杂度的角度来看,从一个原始的解释器生成编译器会更好\footnote%
{也许要经过多个步骤,使用逐渐增强的编译器、解释器乃至汇编器,这种汇编器很可能会
和这里提出的具有 Nil 威力的汇编有些相似;为了方便审计,在能把全程涉及的代码总量
压缩到接近最小的前提之下,中间步骤数应当尽量小。此外从更偏理论的角度上,我觉得
Futamura 投影(本质上是\stress{部分求值})\cupercite{wiki:parteval} 很有趣。}。
类似地,注意到 Nil 的表达力,我们自然可以使用一个能自举的 Nil 编译器
作为生产系统中的基础编译器,而其它编译器(如果仍有必要,
此外其中可能包含 C 编译器)都由它来生成。

在本节中,我们至今考虑的是 Nil 类似于 Lisp 的方面,那么它类似于 C 的方面是怎样
的呢?我认为上节提到的 Dale 是 Nil 中类 C 层的一个可能原型,在这一层之上可以通过
宏构造一个类 Lisp 层,而许多其它语言(如 shell 和 Makefile)将由类 Lisp 层模拟,
它们通过某种像 \hologo{LaTeX} \stress{宏包}那样的机制提供给用户。我设想的另一种
做法基于可移植汇编语言的概念\footnote{因为类 C 层本质上将是一个机器码生成器,
类似的生成器,例如 Chez Scheme 内置的汇编器和 Daniel J.\ Bernstein 的 qhasm,
或许可以作为很有指导意义的参考。}(见第 \ref{sec:wib} 节),它可以称为具有 Nil
威力的汇编:类 C 层将像多数其它语言一样基于类 Lisp 层实现,它将是一个像标记
语言一样的“逆宏”系统,其中提供的基本函数在被调用时将汇编指令发送到输出。

此外正如上节所述,Nil 的类型系统须要比 Lisp 中的更强,就此我认为王垠的基于%
\stress{合约}的设计\cupercite{wangyin2013a, wangyin2013b}是一个很引人深思的
参考;值得指出的是这一设计能帮助我们增强类 C 层的内存安全性,而且甚至可以方便
对程序的形式化验证(见脚注 \ref{fn:formal})。类 Lisp 层和类 C 层之间的外部
函数接口也将是一个问题,但我现在的知识还不足以让我能就此发表意见。最后在结束本节
之前,我须要强调惯性这一非技术问题,它困扰着 Scheme(见脚注 \ref{fn:r6rs})、%
Plan~9(见脚注 \ref{fn:plan9})以及 qmail 和其它优秀(或者不优秀)的软件项目:
除了将不可避免地遇到的技术挑战,Nil 也须要吸引学术界和工业界的注意以产生
足够的动力(后者当然可能完全不关心,但请参考下节);而即使 Nil 获得某种成功,
在丢弃历史包袱的同时适应新的需求对 Nil 也不会是轻松的任务,就像其它项目一样。

\section{走向“超统一”}\label{sec:toe}

正如上节和第 \ref{sec:benefits} 节指出的,Nil 可能是一个即使在
理论意义上也无法实现的语言,它也可能无法从工业界吸引足够的注意,
那么在这样的情况下探索 Nil 是否仍有价值?在这最后一节,我将
基于我在 \parencite{vector2018c} 中作的比喻给出我的回答:
\begin{quoting}
	I guess reconciling Lisp and Unix would be much easier than
	reconciling quantum mechanics and general relativity; and
	it would be, in a perhaps exaggerated sense, as meaningful.
\end{quoting}

理论物理学家和高能物理学家长期追求一种可以统一量子场论和广义相对论的超统一
理论,因为这一理论将为所有底层物理现象提供一个统一的基础。据我所知,我们
似乎并没有可靠地证明人类能够成功地达成这种理论:例如弦论是最为人知晓的
候选理论,它虽然的确符合观测到的事实,但至今似乎仍然远非一个可验证的理论。
但另一方面,我们也没有证明超统一理论无法达成;而且除了超统一理论作为物理
理论的价值,对它的研究还不断地在从纯数学(和理论物理有着深刻的联系)
到土木工程(在建设物理实验用的大型装置时涉及)的许多领域中催生令人
激动的进展。因为这两个原因,超统一理论仍是许多物理学家的毕生追求。

类似地,尽管统一 Lisp 和 C 的努力并不确定能真的产生一种终极语言,对这%
\stress{两种语言最佳协作方式}(最小化系统复杂度、最大化程序员效率)的探索将
必然给我们丰厚的回报:正如我们已经在第 \ref{sec:benefits} 节看到的,即使是在
一种非常原始的形态下,这种统一也能明显地降低 Unix 系统的复杂度;我相信对它们的
进一步简化将吸引更多的人去探索计算机系统的基础层面,就像 Raspberry Pi 重燃人们对
底层开发的热情那样。此外即使我们的努力最终或许不能产生可用于生产环境的基于 Nil
的系统,我也完全确信 Unix 和 Lisp 这两个圈子之间深刻见解的交流将必然产生大量%
\stress{有趣且有意义的副产品},正如第 \ref{sec:worklife} 节中解释的一样,因此我
用分别来自 Rob Pike\cupercite{pike2000}\footnote{讽刺的是,就程序语言而言,他的
话也适用于他自己。} 和 Hal Abelson\cupercite{abelson2008} 的两段引文结束本文档:
% XXX: hackish.
\pagebreak
\begin{quoting}
	Narrowness of experience leads to narrowness of imagination.
\end{quoting}
\colskipc
\begin{quoting}
	If you \emph{[carefully study the example interpreters in this book]},
	you will change your view of your programming, and your view of
	yourself as a programmer.  You'll come to see yourself as a designer
	of languages rather than only a user of languages, as a person
	who chooses the rules by which languages are put together, rather
	than only a follower of rules that other people have chosen.
\end{quoting}



\newpart
\printbibliography[heading = bibintoc, title = 参考资料]
\newpart
\specialsec{跋}\label{sec:afterword}

在开始学习 Linux 之后不久,我购买了《Shell 脚本学习指南》一书,这本书开启了
我第一次热情投入的正式编程学习经历,而这一热情正是源于第 \ref{sec:shell}
节所述的在书中用于演示 Unix 哲学威力的词频统计程序。(注意这和第
\ref{sec:cognitive} 节中所述的 V.~I.\ Arnold 对 L.~A.\ Tumarkin
教学方式所产生反应的相似性。)尽管这本书自始至终贯彻 Unix 哲学(书中称为
“软件工具哲学”),我直到 2009 年春在旁听本校曹东刚老师的《Linux 程序设计环境》
课程时才开始有意识地遵循 Unix 哲学,因为这一课程中对其进行了正式的介绍和强调。

我的 Linux 经历本来是比较平静的,直到 systemd 的出现:2011--2012 年,systemd
开发者推行了让 logind 成为 GNOME 3 的依赖\cupercite{poettering2011a},以及让
udev 并入 systemd\cupercite{sievers2012} 的计划;从此开始的激烈纷争和论战直至
近两年才开始稍有缓和\cupercite{slashdot:systemd},而我也明确希望本文档能加快
systemd 和与之关联的专有式手段走向消亡的速度。“塞翁失马,焉知非福”:在寻找
systemd 替代品的过程中,我在 2014 年夏注意到了 s6 项目,而学习 Daniel
J.\ Bernstein 风格软件的设计和实现被证明是对 Unix 哲学实践的宝贵练习。
在此过程中,“Unix 哲学是否已经过时?”和“Unix 哲学的社会价值是什么?”
这两个问题开始萦绕在我心头,它们最终形成了本文档的前两部分。

2018 年春,在偶然发现 scsh\cupercite{angelbeats2018} 的一年之后,
我在考虑 C 语言缺乏表达力的问题时设想了一种有些像 C 但基于 S 表达式的
语言\cupercite{vector2018b},尽管我可能早在 2012 年时就开始知道 Lisp
有多么优雅,而这主要归功于王垠的文章(尽管他在中文开源界内争议很大,
但是他常能提出很有意思的论点)。启发我寻求上述类 C 语言的问题可以追溯到
更早,即 2016 年初我开始考虑是否能设计一种结合 Bernstein chainloading 的类
shell 语言\cupercite{vector2016a}时。最终我在 2018 年春想到了一个可实际操作
的极简主义方案\cupercite{vector2018c}来实现这样一种语言,并同时产生了
“统一 Lisp / C”的想法,而这一想法经过 Gentoo 论坛上的一系列
讨论\cupercite{stevel2018}成为了本文档的第三部分。

在结束本文档之前,我要感谢北京大学 Linux 俱乐部和 skarnet 软件社区,没有它们
我很可能只会成为又一个对 Linux 略知一二的业余程序员,而本文档也根本不可能问世。
我也要感谢整个开源社区,其中包括 Unix 圈、Lisp 圈以及其它社区,包含支持和反对
systemd 的人,因为它提供了丰富而多样的技术资料和观点,这是一份无价的礼物。
受曹禺所著《雷雨》中背景音乐设定和 Donald Knuth 所著《The \hologo{TeX}book》%
《The \hologo{METAFONT}book》中附录的影响,我有意在本文档中
模仿了 J.~S.\ Bach《b 小调弥撒》的篇章结构。
\end{document}

